\documentclass[twoside,a4paper,12pt]{article}

% Packages required by doxygen
\usepackage[export]{adjustbox} % also loads graphicx
\usepackage{graphicx}
\usepackage[utf8]{inputenc}
\usepackage{multicol}
\usepackage{multirow}
\usepackage{makeidx}
\usepackage[table]{xcolor}

% NLS support packages
\usepackage[T2A]{fontenc}
\usepackage[russian]{babel}

% Font selection
\usepackage[T1]{fontenc}
\usepackage[scaled=.90]{helvet}
\usepackage{courier}
\usepackage{amssymb}
\usepackage{sectsty}

% Page & text layout
\usepackage{geometry}
\geometry{%
  a4paper,%
  top=2.5cm,%
  bottom=2.5cm,%
  left=2.5cm,%
  right=2.5cm%
}
%\setlength{\emergencystretch}{15pt}
\setlength{\parindent}{0cm}
\setlength{\parskip}{0.2cm}

% Headers & footers
\usepackage{fancyhdr}
\pagestyle{fancyplain}
\fancyhead[LE]{\fancyplain{}{\bfseries\thepage}}
\fancyhead[CE]{\fancyplain{}{\scriptsize\textbf{RU.17701729.503200 ТЗ 01-1-ЛУ}}}
\fancyhead[RE]{\fancyplain{}{\bfseries\leftmark}}
\fancyhead[LO]{\fancyplain{}{\bfseries\rightmark}}
\fancyhead[CO]{\fancyplain{}{\scriptsize\textbf{RU.17701729.503200 ТЗ 01-1-ЛУ}}}
\fancyhead[RO]{\fancyplain{}{\bfseries\thepage}}
\fancyfoot[LE]{\fancyplain{}{}}
\fancyfoot[CE]{\fancyplain{}{}}
\fancyfoot[RE]{\fancyplain{}{}}
\fancyfoot[LO]{\fancyplain{}{}}
\fancyfoot[CO]{\fancyplain{}{}}
\fancyfoot[RO]{\fancyplain{}{}}

% Indices & bibliography
\usepackage{natbib}
\usepackage[titles]{tocloft}
\setcounter{tocdepth}{3}
\setcounter{secnumdepth}{5}
\makeindex

% Packages for text layout in normal mode
% \usepackage[parfill]{parskip} % автоматом делает пустые линии между параграфами, там где они есть в тексте
% \usepackage{indentfirst} % indent even in first paragraph
\usepackage{setspace}	 % controls space between lines
\setstretch{1} % space between lines
\setlength\parindent{0.9cm} % size of indent for every paragraph
\usepackage{csquotes}% превратить " " в красивые двойные кавычки
\MakeOuterQuote{"}

% Custom commands


% Custom packages
\usepackage{pdfpages}


%===== C O N T E N T S =====

\begin{document}

% Titlepage & ToC
\pagenumbering{roman}
\begin{titlepage}
\begin{center}
\vspace*{0.7cm}
{\large ПРАВИТЕЛЬСТВО РОССИЙСКОЙ ФЕДЕРАЦИИ \\
НАЦИОНАЛЬНЫЙ ИССЛЕДОВАТЕЛЬСКИЙ УНИВЕРСИТЕТ \\
«ВЫСШАЯ ШКОЛА ЭКОНОМИКИ» }\\
\vspace*{0.2cm}
Факультет компьютерных наук \\
\smallskip
{\small Департамент программнoй инженерии \\
}
\vspace{1cm}

\begin{multicols}{2}
{\large СОГЛАСОВАНО} \\
Глава департамента \\
програмной инженерии \\
факультета компьютерных наук \\
кандидат компьютерных наук \\
\medskip

{\large УТВЕРЖДАЮ} \\
Академический руководитель \\
образовательной программы  \\
«Програмная инженерия» \\
\medskip

\end{multicols}

\vspace{3cm}

{\textbf{ПРОГРАММА СКЕЛЕТНАЯ АНИМАЦИЯ}} \\
\medskip
\textbf{Техническое задание} \\
\medskip
{\textbf{
ЛИСТ УТВЕРЖДЕНИЯ \\
\medskip
RU.17701729.503200 ТЗ 01-1-ЛУ
}} \\

\vfill
\begin{flushright}
Исполнитель \\
Студент группы БПИ 151 НИУ ВШЭ \\
Абрамов А.M. \\
\end{flushright}
\vspace{3cm}

Москва \number\year \\
\end{center}
\end{titlepage}


% second page which is 80% like titlepage
% second page which is like titlepage but a little bit different
\newpage
\vspace*{3cm}
\begin{center}
{ПРОГРАММА СКЕЛЕТНАЯ АНИМАЦИЯ} \\
\medskip
{\small\textbf{Техническое задание}} \\
\medskip
{\textbf{ЛИСТ УТВЕРЖДЕНИЯ} \\
\medskip
RU.17701729.500900 ТЗ 01-1-ЛУ} \\
\bigskip
\bigskip
\textbf{Листов 16}
\end{center}


% some water filling text, that is pointless but adds text
\newpage

\begin{center}
{\large АННОТАЦИЯ}
\end{center}

Настоящий документ представляет собой техническое задание для разработки приложения реализации алгоритма скелетной анимации. Данный документ составлен в соответствии с ГОСТ. В документе содержатся следующие разделы: «Введение», «Основания для разработки», «Назначение разработки», «Требования к программе», «Технико-экономические характеристики», «Стадии и этапы разработки», «Порядок контроля и приемки».

В разделе «Введение» содержится информация о намиеновании и краткой характеристике разрабатываемого приложения.

В разделе «Основания для разработки» содержится информация о документах, на основании которых ведется разработка настоящего приложения, а так же наименование темы разработки.

В разделе «Назначение разработки» содержится информация о функциональном и эксплуатационном назначении разрабатываемого прилоежния.

В разделе «Требования к программному изделию» содержится информация о требованиях к функциональным характеристикам, требованиях к надежности, условиях эксплуатации, требованиях к составу и параметрам технических средств, требования к информационной и программной совместимости.

В разделе «Требования к программной документации» содержится информация о требованиях, в соответствии с которыми должна выполняться разработка программной документации приложения.

В разделе «Технико-экономические показатели» содержится информация об ориентировочной экономической эффективности и ожидаемой годовой потребности, экономических преимуществах разработки по сравнению с лучшими зарубежными и отечественными аналогами.

В разделе «Стадии и этапы разработки» содержится информация о необходимых стадиях разработки, этапах и содержании работ. Присутствует информация о сроках разработки и исполнителях.

В разделе «Порядок контроля и приемки» содержится подробная информация о видах испытаний, которые будут применены к данному приложению, а так же общих требованиях к приемке работ.

Перед прочтением настоящего документа рекомендуется ознакомиться со списком терминов, для предотвращения непонятных моментов.



\newpage
\tableofcontents
\pagenumbering{arabic}

% --- add my custom headers ---
\section{Введение}
\subsection{Наименование}
Наименование: «Программа скелетная анимация». Англ.: «Program of Skeletal Animation».


\subsection{Краткая характеристика}
    Цель работы - реализовать систему скелетной анимации. 
    В основные задачи работы входит загрузка анимации из этих файлов, рассчет промежуточных кадров анимации и воспроизведение анимации на экране средствами OpenGL. 
    Также программа предоставляет пользовтелю возможность рассмотреть анимацию с разных ракурсов, перейти к любому моменту времени в анимации и просмотреть иерархию костей и модели.
В состав работы также входит создание демонстрационных исходных данных (файлов) для данной системы.


\section{Основания для разработки}
\section{Назначение разработки}
\section{Требования к програмному изделию}
\section{Требования к програмной документации}
\section{Технико-экономические показатели}
\section{Стадии и этапы разработки}
\section{Порядок контроля и приемки}
\section{Приложение 1. Терминология}
\section{Приложение 2. Список используемой литературы}
\section{Приложение 3. Изображение пользовательского интерфейса.}

% Index
\newpage
\phantomsection
\addcontentsline{toc}{section}{Алфавитный указатель}
\printindex

\end{document}
