%=========================================
\subsection{Параметру технических средств, используемых во время испытаний}
Для испытания программы необходимо учесть следующие системные требования:
\begin{my_enumerate}
\item Компьютер, оснащенный:
    \begin{my_enumerate}
    \item Обязательно 64-разрядный (x64) процессор с тактовой частотой 1 гигагерц (ГГц) или выше;
    \item 256 мегабайт (МБ) оперативной памяти (ОЗУ);
    \item 2 ГБ (для 64-разрядной системы) пространства на жестком диске;
    \item графическое устройство OpenGL с драйвером версии 3.1 или выше.
    \end{my_enumerate}
\item Монитор
\item Видеокарта
\item Мышь
\item Клавиатура
\end{my_enumerate}


%=========================================
\subsection{Программные средства, необходимые для проведения испытаний}
Приложению необходим компьютер с поддержкой OpenGL версии не менее 3.1. 64-битная операционная система Windows 7 или более поздняя версия Windows. Должен быть установлен .NET Framework версии не ниже 4.5.1, а также библиотеки Assimp версии не ниже 3.1 и OpenTK версии не ниже 1.1.4

%=========================================
\subsection{Порядок проведения испытаний}
Испытания должны проводиться в следующем порядке:
\begin{my_enumerate}
\item Проверка требований к документации.
\item Проверка требований к интерфейсу.
\item Проверка требований к функциональным возможностям программы.
\item Проверка требований надежности.
\end{my_enumerate}


%=========================================
\subsection{Условия проведения испытаний}

\subsubsection{Требования к численности и калификации персонала}
Для испытания программы требуется один оператор. Оператор программы должен иметь образование не ниже среднего общего и обладать базовыми знаниями следующих понятий из линейной алгебры, программирования и 3-х мерного моделирования: вектор, кватернион, матрица направляющих косинусов, меш (англ. mesh), кость, корневая вершина (англ. root node).




