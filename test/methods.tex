Испытания представляют собой процесс установления соответствия программы и
программной документации заданным требованиям.

\subsubsection{Проверка требований к документации}
Проверяеться наличие всех документов перечисленных в пyнкте 4.1 данного документа и их соответствие ГОСТ.


\subsection{Проверка требований к интерфейсу}

\begin{figure}[h!]
    \centering
    \framebox[1cm][l]{never mind, so am I}
    %\makebox[2pt][c]{My image}
    %\includegraphics[width=0.8\textwidth]{image.png}
    \caption{Awesome Image}
    \label{fig:awesome_image}
\end{figure}

Интерфейс обладает шкалой времени, панелью типа TreeView и объектом типа GLControl для отрисовки трехмерных элементов из библиотеки OpenTK.



\subsection{Проверка требований к функциональным характеристикам}
Для загрузки данных из формата коллада (collada или .dae) необходимо выбрать его в меню:

\begin{figure}[h!]
    \centering
    \framebox[1cm][l]{never mind, so am I}
    %\makebox[2pt][c]{My image}
    %\includegraphics[width=0.8\textwidth]{image.png}
    \caption{Awesome Image}
    \label{fig:awesome_image}
\end{figure}



%=================
После загрузки файла видно что его имя добавилось в список недавно открытых файлов "Recent Files".

\begin{figure}[h!]
    \centering
    \framebox[1cm][l]{never mind, so am I}
    %\makebox[2pt][c]{My image}
    %\includegraphics[width=0.8\textwidth]{image.png}
    \caption{Awesome Image}
    \label{fig:awesome_image}
\end{figure}



%=================
У пользователя имеется возможность изменять ракурс и приближение камеры при помощи мышки. Структура загруженных данных отображена в виде дерева на панели справа. Также выделенная кость подсвеченна ярко-синим цветом.

\begin{figure}[h!]
    \centering
    \framebox[1cm][l]{never mind, so am I}
    %\makebox[2pt][c]{My image}
    %\includegraphics[width=0.8\textwidth]{image.png}
    \caption{Awesome Image}
    \label{fig:awesome_image}
\end{figure}



%=================
Элемент ScrollBar показывает текущий момент в анимации и предоставляет 
возможность перейти к любому моменту времени. Также есть панель для настоек работы программы позволяющая изменять следующие параметры:
\begin{my_enumerate}
\item Выбор между двумя видами камер в OpenGL (скованой орбитой и свободной).
\item Вкл./Выкл. воспроизведения анимации.
\item Вкл./Выкл. отрисовку учитывая нормали данной модели.
\item Вкл./Выкл. отрисовки материала данной модели.
\item Отрисовка всех костей.
\end{my_enumerate}


\begin{figure}[h!]
    \centering
    \framebox[1cm][l]{never mind, so am I}
    %\makebox[2pt][c]{My image}
    %\includegraphics[width=0.8\textwidth]{image.png}
    \caption{Awesome Image}
    \label{fig:awesome_image}
\end{figure}


\bigskip

%=================
Поддерживатся изменение размеров окна приложения, без изменения соотношения проекции OpenGL.

\begin{figure}[h!]
    \centering
    \framebox[1cm][l]{never mind, so am I}
    %\makebox[2pt][c]{My image}
    %\includegraphics[width=0.8\textwidth]{image.png}
    \caption{Awesome Image}
    \label{fig:awesome_image}
\end{figure}



\subsection{Проверка требований к надежности}
Оператор должен поочередно воспользоваться всеми функциями программы и убедиться что они не приводят к ее аварийному завершению.
