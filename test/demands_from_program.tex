

%=========================================
\subsection{Требования к функциональным характеристикам}
\subsubsection{Требования к составу выполняемых функций}
\begin{my_enumerate}
\item Чтение данных из формата коллада (collada или .dae).
\item Возможность изменять положение и ракурс камеры в OpenGL.
\item Поддержка двух видов камер в OpenGL, первый вид это камера движение которой сковано орбитой вокруг модели и другой тип это камера двигающаяся совершенно свободно.
\item Возможность перейти к любому моменту времени в анимации.
\item Последовательно воспроизведение анимации на экране.
\item Возможность передвигать камеру во время анимации.
\item Вкл./Выкл. отрисовку учитывая нормали данной модели.
\item Вкл./Выкл. отрисовки материала данной модели.
\item Возможность изменять положение модели при работе с камерой скованной орбитой.
\item Отображение информации об объектах типа кости и меш, и реляционные связи между ними в виде дерева.
\item Отрисовка всех костей. Отрисовка должна производиться поверх модели.
\item Подсветка отдельных костей выбранных пользователем.
\item Изменение параметров проигрывания анимации, а именно времени, также возможность проиграть в обратную сторону последний интервал между ключевыми фреймами.
\item Анимация должна автоматически запускаться заново после отображения последнего ключевого фрейма.
\end{my_enumerate}

\subsubsection{Требования к организации входных и выходных данных}
\begin{my_enumerate}
\item Входными данными для программы являются файлы  созданные и экспортированные из пакета 3-х мерного моделлирования (примерами таких пакетов являются Blender, Maya, Cinema4D).
\item Из-за огромного колличества форматов для описания геометрических, объектных и анимационных данных, поддерживать их все не представляется возможным. Поэтому программа должна работать только с форматом Collada (.dae).
\item Для успешной загрузки в программу созданна сцена должна содержать:
\begin{my_enumerate}
\item Oдин меш с соответствующим ему тегом <Name="Mesh"> файле коллада.
\item Один трэк анимации.
\item Один скелет связанный с мешем, и с соответствующим ему тегом <Name="Armature"> в файле коллада.
\end{my_enumerate}
\item Пользователь должен иметь возможность модифицировать следующие входные данные в ходе работы программы:
\begin{my_enumerate}
\item Время для которог надо отобразить анимацию.
\item Положение/ориентация модели в OpenGL.
\item Модификаторы используемые при отрисовке модели и костей.
\item Положение/ракурс камеры в OpenGL.
\end{my_enumerate}
\item Выходные данные для программы это отображение на экране.
\end{my_enumerate}

\subsubsection{Прочие требования}
\begin{enumerate}
\item Приложение должно вести список недавно открытых файлов.
\item Должно поддерживаться изменение размеров окна приложения, без изменения соотношения проекции OpenGL.
\end{enumerate}

%=========================================
\subsection{Требования к временным характеристикам}
\begin{enumerate}
\item Задержка между кадрами отриванными на экране не должна превышать 0.1 секунд для моделей составленных из менее чем $2,000,000$ треугольников, $15$ костей и  $1,500,000$ вершин.
\end{enumerate}


%=========================================
\subsection{Требования к интерфейсу}
\begin{enumerate}
\item Интерфейс должен обладать шкалой времени для выбора момента времени для анимации.
\item Должна присутствовать панель типа TreeView для отображение иерархии в данном файле.
\item Для отрисовки 3-х мерных элементов должен использоваться элемент GLControl из библиотеки OpenTK.
\end{enumerate}


%=========================================
\subsection{Требования к надежности}
\subsubsection{Обеспечение устойчивого функционирования программы}
Программа не должна вне зависимости от входных данных или действий оператора завершатся аварийно. При некорректно введенных параметрах пользователю должно отображаться сообщение об ошибке.
\subsubsection{Время восстановления после отказа}
Требования к восстановлению после отказа не предъявляются.
\subsubsection{Отказы из-за некорректных действий оператора}
Требования к отказу из-за некорректных действий оператора не предъявляются.
