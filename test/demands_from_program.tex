

%=========================================
\subsection{Требования к функциональным характеристикам}
\subsubsection{Требования к составу выполняемых функций}
\begin{my_enumerate}
\item Чтение данных из формата collada для хранения трех мерных моделей.
\item Изменение положения и ракурса камеры в OpenGL.
\item Загрузка не более одной сцены одновременно.
\item Поддержка двух видов камер в OpenGL, первый вид это камера движение которой сковано орбитой вокруг модели и другой тип это камера двигающаяся совершенно свободно.
\item Переход к любому моменту времени в анимации.
\item Воспроизведение анимации на экране.
\item Передвижение камеры во время анимации.
\item Включение и выключение отрисовки с учетом нормалей.
\item Включение и выключение отрисовки с учетом характеристик материала.
\item Изменение положения модели.
\item Отображение информации об объектах типа кости и меш, и реляционные связи между ними. 
\item Отрисовка всех костей скелета. Отрисовка должна производиться поверх модели.
\item Подсветка отдельных костей выбранных пользователем.
\item Изменение параметров проигрывания анимации, а именно времени, также возможность проиграть в обратную сторону последний интервал между ключевыми фреймами.
\item Перезапуск анимации с начального кадра после отображения последнего ключевого фрейма.
\end{my_enumerate}

\subsubsection{Требования к организации входных и выходных данных}
Входными данными для программы являются файл анимации, а также (для обеспечения взаимодействия с пользователем) клавиатура и мышь. Входные данные могут быть созданны в любом пакете для трех мерного моделирования. Примером такого пакета является Blender (https://www.blender.org/, разработчик: некоммерческая организация Blender Foundation).

\begin{my_enumerate}
\item Из-за огромного количества форматов для описания анимационных данных, поддерживать их все не представляется возможным. Поэтому программа должна работать только с форматом collada (.dae).
\item Файл должен содержать одну модель, один трэк анимации и один скелет для модели.
\item Пользователь должен иметь возможность модифицировать следующие входные данные в ходе работы программы:
\begin{my_enumerate}
\item Время для которого надо отобразить анимацию.
\item Положение/ориентация модели в OpenGL.
\item Затемнение модели (нормали, материал).
\item Положение/ракурс камеры в OpenGL.
\end{my_enumerate}
\end{my_enumerate}

\medskip
Выходными данными для программы является отображение на экране.

\subsubsection{Прочие требования}
\begin{enumerate}
\item Хранение списка недавно открытых файлов.
\item Поддержка изменения размеров окна без искажения проекции OpenGL.
\end{enumerate}

%=========================================
\subsection{Требования к временным характеристикам}
\begin{enumerate}
\item Задержка между кадрами отрисованными на экране не должна превышать 0.1 секунд для моделей составленных из не болле чем $2,000,000$ треугольников, $15$ костей и $1,500,000$ вершин.
\end{enumerate}


%=========================================
\subsection{Требования к интерфейсу}
Интерфейс должен удовлетворать схеме в приложении 2.

%=========================================
\subsection{Требования к надежности}
\subsubsection{Обеспечение устойчивого функционирования программы}
Программа не должна вне зависимости от входных данных или действий оператора завершатся аварийно. При некорректно введенных параметрах пользователю должно отображаться сообщение об ошибке.
\subsubsection{Время восстановления после отказа}
Требования к восстановлению после отказа не предъявляются.
\subsubsection{Отказы из-за некорректных действий оператора}
В случае открытия файла, не соответствующему требованиям ко входным данным, пользователю должно отображаться сообщение об ошибке.
