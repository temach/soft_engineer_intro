\begin{scriptsize}

\subsection{ActionState}

\begin{verbatim}
using System;
using System.Collections.Generic;
using System.Linq;
using System.Text;
using System.Threading.Tasks;
using System.Windows.Forms;
using System.Drawing.Drawing2D;
using System.Drawing;
using System.IO;        // for MemoryStream
using System.Reflection;
using System.Diagnostics;
using Assimp;
using System.ComponentModel;
using System.Runtime.CompilerServices;
using Assimp.Configs;
using d2d = System.Drawing.Drawing2D;
using tk = OpenTK;
using Matrix4 = OpenTK.Matrix4;

namespace WinFormAnimation2D
{

    /// <summary>
    /// This class knows what argumets to pass to NodeInterpolator.
    /// </summary>
    class ActionState : BaseForEventDriven
    {

        public Animation _action;

        // owner = only used to get the global transform matrix for root bone
        public Entity _owner;
        public Matrix4 GlobalTransform
        {
            get {
                Debug.Assert(_owner != null);
                return _owner._transform._matrix;
            }
        }

        // index of keyframe maps to its time in ticks
        public List<double> KeyframeTimes;
        public int KeyframeCount
        {
            get { return KeyframeTimes.Count; }
        }
        public int FinalKeyframe
        {
            get { return KeyframeCount - 1; }
        }

        public string Name
        {
            get { return _action.Name; }
        }
        /// Duration of animation.
        public double TotalDurationSeconds
        {
            get { return _action.DurationInTicks * _action.TicksPerSecond; }
        }
        public double TotalDurationTicks
        {
            get { return _action.DurationInTicks; }
        }

        /// position of the time cursor in ticks of animation.
        public double TimeCursorInTicks
        {
            get
            {
                double interval_ticks = (KeyframeTimes[TargetKeyframe] - KeyframeTimes[OriginKeyframe]);
                return KeyframeTimes[OriginKeyframe] + interval_ticks * KfBlend;
            }
        }

        public double IntervalLengthMilliseconds
        {
            get
            {
                double interval_ticks = Math.Abs(KeyframeTimes[TargetKeyframe] - KeyframeTimes[OriginKeyframe]);
                double interval_seconds = interval_ticks * _action.TicksPerSecond;
                return interval_seconds * 1000.0;
           }
        }

        /// TickPerSec can be used to change speed.
        private double _tps;
        public double TickPerSec
        {
            get { return _tps; }
            set { _tps = value; }
        }

        /// Start or origin keyframe
        private int _origin_keyframe;
        public int OriginKeyframe
        {
            get { return _origin_keyframe; }
            set
            {
                // Note: frame is strictly less than KeyframeCount
                if (0 <= value && value < KeyframeCount)
                {
                     _origin_keyframe = value;
                }
            }
        }

        /// End or target keyframe
        private int _target_keyframe;
        public int TargetKeyframe
        {
            get { return _target_keyframe; }
            set
            {
                // Note: frame is strictly less than KeyframeCount
                if (0 <= value && value < KeyframeCount)
                {
                     _target_keyframe = value;
                }
            }
        }

        /// Blend value between 0.0 - 1.0, how much in between two keyframes are we
        private double _kf_blend;
        public double KfBlend
        {
            get { return _kf_blend; }
            set
            {
                _kf_blend = Math.Min(Math.Max(0, value), 1.0);
                NotifyPropertyChanged();
            }
        }

        /// Automatically play the animation again after it has timed out.
        public bool _loop;
        public bool Loop
        {
            get {
                return _loop;
            }
            set { 
                _loop = value;
                if (_loop)
                {
                    SetTime(0);
                }
                NotifyPropertyChanged();
            }
        }

        public ActionState(Animation action)
        {
            SetCurrentAction(action);
        }

        public void NextInterval()
        {
            OriginKeyframe = Loop ? TargetKeyframe % (FinalKeyframe) : TargetKeyframe;
            TargetKeyframe = OriginKeyframe + 1;
            KfBlend = 0.0;
        }

        public void ReverseInterval()
        {
            OriginKeyframe = TargetKeyframe;
            TargetKeyframe -= 1;
            KfBlend = 1.0 - KfBlend;
        }

        /// Change the animation track. If there is more than one. We don't support this yet.
        public void SetCurrentAction(Animation action)
        {
            _action = action;
            _tps = action.TicksPerSecond;
            KfBlend = 0; 
            // Keyframe times must be initialised before Origin/Target Keyframes
            KeyframeTimes = _action.NodeAnimationChannels[0].PositionKeys.Select(vk => vk.Time).ToList();
            OriginKeyframe = 0;
            TargetKeyframe = 0;
        }

        public int FindStartFrameAtTime(double time_ticks)
        {
            Debug.Assert(time_ticks >= 0);
            // sometimes first time is non zero (e.g. 0.045)
            if (time_ticks <= KeyframeTimes[0])
            {
                return 0;
            }
            for (int i = 1; i < KeyframeCount; i++)
            {
                if (time_ticks < KeyframeTimes[i])
                {
                    return i - 1;
                }
            }
            // return last frame if not found (because of numerical inaccuracies?)
            return KeyframeCount - 1;
        }

        /// Set the current time for the animation.
        /// Note: all the calculations here are done in ticks.
        public void SetTime(double time_seconds)
        {            
            double time_ticks = time_seconds * TickPerSec;
            // when time overflows we loop by default
            double time = time_ticks % TotalDurationTicks;
            int start_frame = FindStartFrameAtTime(time_seconds);
            int end_frame = (start_frame + 1) % KeyframeCount;
            double delta_ticks = KeyframeTimes[end_frame] - KeyframeTimes[start_frame];
            // when we looped the animation
            if (delta_ticks < 0.0)
            {
                delta_ticks += TotalDurationTicks;
            }
            double blend = (time - KeyframeTimes[start_frame]) / delta_ticks;
            // assign results
            OriginKeyframe = start_frame;
            TargetKeyframe = end_frame;
            KfBlend = blend;
        }

    }
}

\end{verbatim}

\subsection{BoneNode.cs}

\begin{verbatim}
using System;
using System.Collections.Generic;
using System.Linq;
using System.Text;
using System.Threading.Tasks;
using System.Windows.Forms;
using System.Drawing.Drawing2D;
using System.Drawing;
using System.IO;        // for MemoryStream
using System.Reflection;
using System.Diagnostics;
using Assimp;
using System.ComponentModel;
using System.Runtime.CompilerServices;
using Assimp.Configs;
using d2d = System.Drawing.Drawing2D;
using tk = OpenTK;
using Matrix4 = OpenTK.Matrix4;

namespace WinFormAnimation2D
{

    // Node with extended properties
    class BoneNode
    {
        public Node _inner;
        public Matrix4 GlobalTransform;
        public Matrix4x4 GlobTrans
        {
            get { return GlobalTransform.eToAssimp(); }
            set { GlobalTransform = value.eToOpenTK(); }
        }
        public Matrix4 LocalTransform;
        public Matrix4x4 LocTrans
        {
            get { return LocalTransform.eToAssimp(); }
            set {  LocalTransform = value.eToOpenTK(); }
        }

        public BoneNode Parent;
        public List<BoneNode> Children;

        public BoneNode(Node assimp_node)
        {
            _inner = assimp_node;
            Children = new List<BoneNode>(assimp_node.ChildCount);
        }

    }


\end{verbatim}
\subsection{MatrixExtensions.cs}
\begin{verbatim}
using System;
using System.Collections.Generic;
using System.Linq;
using System.Text;
using System.Threading.Tasks;
using ai = Assimp;
using Assimp.Configs;
using System.Windows.Forms;
using System.Drawing;
using d2d = System.Drawing.Drawing2D;
using tk = OpenTK;

namespace WinFormAnimation2D
{
    static class AssimpMatrixExtensions
    {

        /// <summary>
        /// Transform a direction vector by the given Matrix. Note: this is for assimp 
        /// matrix which is row major.
        /// </summary>
        /// <param name="vec">The vector to transform</param>
        /// <param name="mat">The desired transformation</param>
        /// <param name="result">The transformed vector</param>
        public static ai.Vector3D eTransformVector(this ai.Matrix4x4 mat, ai.Vector3D vec)
        {
            return new ai.Vector3D
            {
                X = vec.X * mat.A1
                    + vec.Y * mat.B1
                    + vec.Z * mat.C1
                    + mat.A4,
                Y = vec.X * mat.A2
                    + vec.Y * mat.B2
                    + vec.Z * mat.C2
                    + mat.B4,
                Z = vec.X * mat.A3
                    + vec.Y * mat.B3
                    + vec.Z * mat.C3
                    + mat.C4
            };
        }

        /// <summary>
        /// Convert 4x4 Assimp matrix to OpenTK matrix.
        /// Will be a very useful function becasue Assimp 
        /// matrices are very limited.
        /// </summary>
        /// <param name="m"></param>
        /// <returns></returns>
        public static tk.Matrix4 eToOpenTK(this ai.Matrix4x4 m)
        {
            return new tk.Matrix4
            {
                M11 = m.A1,
                M12 = m.B1,
                M13 = m.C1,
                M14 = m.D1,
                M21 = m.A2,
                M22 = m.B2,
                M23 = m.C2,
                M24 = m.D2,
                M31 = m.A3,
                M32 = m.B3,
                M33 = m.C3,
                M34 = m.D3,
                M41 = m.A4,
                M42 = m.B4,
                M43 = m.C4,
                M44 = m.D4
            };
        }

        /// <summary>
        /// Convert assimp 4 by 4 matrix into 3 by 2 matrix from System.Drawing.Drawing2D and use it
        /// for drawing with Graphics object.
        /// </summary>
        public static d2d.Matrix eTo3x2(this ai.Matrix4x4 m)
        {
            return new d2d.Matrix(m.A1, m.B1, m.A2, m.B2, m.A4, m.B4);
            // return new draw2D.Matrix(m[0, 0], m[1, 0], m[0, 1], m[1, 1], m[0, 3], m[1, 3]);
        }

        public static ai.Matrix4x4 eSnapTranslation(this ai.Matrix4x4 m, ai.Vector3D vec)
        {
            throw new NotImplementedException("Either make this method for assimp use, or change to OpenTK matrices!");
        }

        public static ai.Vector3D eGetTranslation(this ai.Matrix4x4 m)
        {
            return new ai.Vector3D(m.A4, m.B4, m.C4);
        }

    }
}

\end{verbatim}
\subsection{QuaternionExtensions.cs}
\begin{verbatim}
using System;
using System.Collections.Generic;
using System.Linq;
using System.Text;
using System.Threading.Tasks;
using System.Windows.Forms;
using System.Drawing.Drawing2D;
using System.Drawing;
using System.IO;        // for MemoryStream
using System.Reflection;
using System.Diagnostics;
using Assimp;
using Assimp.Configs;
using d2d = System.Drawing.Drawing2D;
using tk = OpenTK;

namespace WinFormAnimation2D
{
    static class AssimpQuaternionExtensions
    {
        public static Matrix4x4 eToMatrix(this Quaternion q)
        {
            float w = q.W, x = q.X, y = q.Y, z = q.Z;
            float xx = 2.0f * x * x;
            float yy = 2.0f * y * y;
            float zz = 2.0f * z * z;
            float xy = 2.0f * x * y;
            float zw = 2.0f * z * w;
            float xz = 2.0f * x * z;
            float yw = 2.0f * y * w;
            float yz = 2.0f * y * z;
            float xw = 2.0f * x * w;
            return new Matrix4x4(1.0f-yy-zz, xy + zw, xz - yw, 0.0f,
                                 xy - zw, 1.0f-xx-zz, yz + xw, 0.0f,
                                 xz + yw, yz - xw, 1.0f-xx-yy, 0.0f,
                                 0.0f, 0.0f, 0.0f, 1.0f);
        }

        public static tk.Quaternion eToOpenTK(this Quaternion q)
        {
            return new tk.Quaternion(q.X, q.Y, q.Z, q.W);
        }
    }
}

\end{verbatim}
\subsection{VectorExtensions.cs}
\begin{verbatim}
using System;
using System.Collections.Generic;
using System.Linq;
using System.Text;
using System.Threading.Tasks;
using ai = Assimp;
using Assimp.Configs;
using System.Windows.Forms;
using System.Drawing;
using d2d = System.Drawing.Drawing2D;
using tk = OpenTK;

namespace WinFormAnimation2D
{
    static class AssimpVectorExtensions
    {

        /// <summary>
        /// Convert assimp 3D vector to 2D System.Drawing.Point
        /// for drawing with Graphics object.
        /// </summary>
        public static Point eToPoint(this ai.Vector3D v)
        {
            return new Point((int)v.X, (int)v.Y);
        }

        /// <summary>
        /// Convert assimp 3D vector to 2D System.Drawing.PointF (floating point)
        /// for drawing with Graphics object.
        /// </summary>
        public static PointF eToPointFloat(this ai.Vector3D v)
        {
            return new PointF(v.X, v.Y);
        }

        /// <summary>
        /// Convert assimp 3D vector to opentk 2D vector.
        /// </summary>
        public static tk.Vector2 eAs2D_OpenTK(this ai.Vector3D v)
        {
            return new tk.Vector2(v.X, v.Y);
        }

        /// <summary>
        /// Convert assimp 3D vector to opentk 3D vector.
        /// </summary>
        public static tk.Vector3 eToOpenTK(this ai.Vector3D v)
        {
            return new tk.Vector3(v.X, v.Y, v.Z);
        }

    }
}

\end{verbatim}
\subsection{CameraDevice.cs}
\begin{verbatim}
using System;
using System.Collections.Generic;
using System.Linq;
using System.Text;
using System.Threading.Tasks;
using OpenTK;
using System.Drawing.Drawing2D;
using System.ComponentModel;
using System.Windows.Forms;
using System.Drawing;
using System.Runtime.CompilerServices;
using System.Diagnostics;

namespace WinFormAnimation2D
{

    enum CamMode
    {
        FreeFly
        , Orbital
    }

    /// <summary>
    /// Maintains camera abstraction. Allows support for orbiting, free fly and even 2D camera.
    /// </summary>
    class CameraDevice
    {
        /// Return the currently active camera mode.
        public CamMode _cam_mode
        {
            get { return Properties.Settings.Default.OrbitingCamera ? CamMode.Orbital : CamMode.FreeFly; }
        }
        public CameraFreeFly3D _3d_freefly;
        public OrbitCameraController _3d_orbital;

        /// Get the translation part of the camera matrix.
        public Vector3 GetTranslation
        {
            get
            { return (_cam_mode == CamMode.Orbital) 
                    ? _3d_orbital.GetTranslation 
                    : _3d_freefly.GetTranslation;
            }
        }

        /// Get the mouse position and calculate the world coordinates based on the screen coordinates.
        public Vector3 ConvertScreen2WorldCoordinates(Point screen_coords)
        {
            return Vector3.Zero;
        }

        /// Constructor
        public CameraDevice(Matrix4 opengl_init_mat)
        {
            _3d_freefly = new CameraFreeFly3D(opengl_init_mat);
            _3d_orbital = new OrbitCameraController();
        }

        /// Get the camera matrix to be uploaded to drawing 2D
        public Matrix4 MatrixToOpenGL()
        {
            return _cam_mode == CamMode.Orbital 
                    ? _3d_orbital.MatrixToOpenGL() 
                    : _3d_freefly.MatrixToOpenGL();
        }

        public void RotateAround(Vector3 axis)
        {
            _3d_freefly.ClockwiseRotateAroundAxis(axis);
            _3d_orbital.MouseMove((int)axis.X, (int)axis.Y);
            _3d_orbital.Scroll(axis.Z);
        }

        /// Respond to mouse events
        public void OnMouseMove(int x, int y)
        {
            _3d_freefly.ProcessMouse(x, y);
            _3d_orbital.MouseMove(x, y);
        }

        /// Zoom in/out of the scene.
        public void Scroll(float scroll)
        {
            _3d_freefly.MoveBy(new Vector3(0, 0, -1 * scroll));
            _3d_orbital.Scroll(scroll);
        }

        // x,y are direction parameters one of {-1, 0, 1}
        public void MoveBy(Vector3 direction)
        {
            _3d_freefly.MoveBy(direction);
            _3d_orbital.Pan(direction.X, direction.Y);
        }

    }

}
\end{verbatim}
\subsection{DrawConfig.cs}
\begin{verbatim}
using System;
using System.Collections.Generic;
using System.Linq;
using System.Text;
using System.Threading.Tasks;
using System.Drawing;

namespace WinFormAnimation2D
{
    /// This class will be passed into the Entity GetSettings() function to make the scene look best.
    class DrawConfig
    {
        // OpenGL settings
        // here is a template:
        /// Enable and disable OpenGL functionallity
        public bool EnableTexture2D = false;
        /// Enable and disable OpenGL functionallity
        public bool EnablePerspectiveCorrectionHint = false;
        /// Enable and disable OpenGL functionallity
        public bool EnableDepthTest = false;
        /// Enable and disable OpenGL functionallity
        public bool EnableFaceCounterClockwise = false;
        /// Enable and disable OpenGL functionallity
        public bool EnableDisplayList = false;
        /// Enable and disable OpenGL functionallity
        public bool EnablePolygonModeFill = false;
        /// Enable and disable OpenGL functionallity
        public bool EnablePolygonModeLine = false;
        /// Enable and disable OpenGL functionallity
        public bool EnableLight = false;

        public bool RenderWireframe = false;
        public bool RenderTextured = true;
        public bool RenderLit = true;

        public Pen DefaultPen = Pens.Gold;
        public Brush DefaultBrush = Brushes.Gold;

        // Font to be used for textual overlays in 3D view (size ~ 12px)
        public readonly Font DefaultFont12;
        // Font to be used for textual overlays in 3D view (size ~ 16px)
        public readonly Font DefaultFont16;

        public DrawConfig()
        {
            DefaultFont12 = new Font(FontFamily.GenericSansSerif, 12);
            DefaultFont16 = new Font(FontFamily.GenericSansSerif, 16);
        }
    }

}

\end{verbatim}
\subsection{Entity.cs}
\begin{verbatim}
using System;
using System.Collections.Generic;
using System.Linq;
using System.Text;
using System.Threading.Tasks;
using Assimp;
using Assimp.Configs;
using System.Windows.Forms;
using System.Drawing.Drawing2D;
using System.Drawing;
using System.IO;        // for MemoryStream
using System.Reflection;
using OpenTK;
using OpenTK.Graphics.OpenGL;
using System.Diagnostics;
using Quaternion = Assimp.Quaternion;

namespace WinFormAnimation2D
{

    /// <summary>
    /// Represents the currently loaded object.
    /// One day we will have lots of these.
    /// </summary>
    class Entity
    {
        public ActionState _action;
        public BoneNode _armature;
        public Node _node;
        public SceneWrapper _scene;
        public Geometry _extra_geometry;
        public DrawConfig _draw_conf;
        public TransformState _transform;
        public Dictionary<int,MeshDraw> _mesh_id2mesh_draw = new Dictionary<int,MeshDraw>();
        public Matrix4 Matrix
        {
            get { return _transform._matrix; }
            set { _transform._matrix = value; }
        }

        public string Name
        {
            get { return _node.Name; }
            set { _node.Name = value; }
        }
        public Vector2 GetTranslation
        {
            get { return Matrix.ExtractTranslation().eTo2D(); }
        }

        // the only public constructor
        // TODO: change the "Node mesh". This should point to MeshDraw object which is unique to each entity.
        public Entity(SceneWrapper sc, Node mesh, BoneNode armature, ActionState state)
        {
            _scene = sc;
            _node = mesh;
            _extra_geometry = new Geometry(sc._inner.Meshes, mesh, armature);
            _armature = armature;
            _action = state;
            _transform = new TransformState(Matrix4.Identity, 10, 17);
        }

        public void UploadMeshVBO(IList<Material> materials)
        {
            InnerMakeMeshDraw(_scene._inner.Meshes, materials);
        }

        // Make a class that will be responsible for managind the buffer lists
        public void InnerMakeMeshDraw(IList<Mesh> meshes, IList<Material> materials)
        {
            for (int i = 0; i < meshes.Count; i++)
            {
                _mesh_id2mesh_draw[i] = new MeshDraw(meshes[i], materials);
            }
        }

        public void RotateBy(double angle_degrees)
        {
            _transform.Rotate(angle_degrees);
        }

        // x,y are direction parameters one of {-1, 0, 1}
        public void MoveBy(int x, int y)
        {
            var translate = _transform.TranslationFromDirection(new Vector3(x, y, 0));
            _transform.ApplyTranslation(translate);
        }

        public bool ContainsPoint(Vector2 p)
        {
            // modify the point so it is in entity space
            Vector3 tmp = new Vector3(p.X, p.Y, 0.0f);
            return _extra_geometry.EntityBorderContainsPoint(tmp.eTo2D());
        }

        /// Render the model stored in EntityScene useing the DrawConfig settings object.
        public void RenderModel(DrawConfig settings)
        {
            _draw_conf = settings;
            if (_draw_conf.EnablePerspectiveCorrectionHint)
            {
                // all are from System.Drawing.Drawing2D.
            }
            // second pass: render with this matrix
            RecursiveRenderSystemDrawing(_node);
            // apply the matrix to graphics just to draw the rectangle
            // TODO: we should just transform the border according to the RecursiveTransformVertices
            RenderBoundingBoxes(_extra_geometry);
        }

        // Render the scene.
        // each vertex at most one bone policy
        private void RecursiveRenderSystemDrawing(Node nd)
        {
            foreach(int mesh_id in nd.MeshIndices)
            {
                MeshDraw mesh_draw = _mesh_id2mesh_draw[mesh_id];
                mesh_draw.RenderVBO();
            }
            foreach (Node child in nd.Children)
            {
                RecursiveRenderSystemDrawing(child);
            }
        }

        public void RenderBoundingBoxes(Geometry geom)
        {
            foreach (var aabb in geom._mesh_id2box.Values)
            {
                if (Properties.Settings.Default.RenderAllMeshBounds)
                {
                    aabb.Render();
                }
            }
        }

        /// Deform the model vertices to align with the skeleton.
        public void UpdateModel(double dt_ms)
        {
            // first pass: calculate a matrix for each vertex
            RecursiveCalculateVertexTransform(_node, Matrix4.Identity.eToAssimp());
            RecursiveTransformVertices(_node);
        }

        // First pass: calculate the transofmration matrix for each vertex
        // here we must associate a matrix with each bone (maybe with each vertex_id??)
        // then we multiply the current_bone matrix with the one we had before 
        // (perhaps it was identity, perhaps it was already some matrix (if 
        // the bone influences many vertices) )
        // then we store this multiplied matrix.
        // in the render function we get a vertex_id, so we can find the matrix to apply 
        // to the vertex, then we send the vertex to OpenGL
        /// Find the appropriate matrix to apply to the given vertex.
        public void RecursiveCalculateVertexTransform(Node nd, Matrix4x4 current)
        {
            Matrix4x4 current_node = current * nd.Transform;
            foreach(int mesh_id in nd.MeshIndices)
            {
                Mesh cur_mesh = _scene._inner.Meshes[mesh_id];
                MeshDraw mesh_draw = _mesh_id2mesh_draw[mesh_id];
                foreach (Bone bone in cur_mesh.Bones)
                {
                    // a bone transform is more than by what we need to trasnform the model
                    BoneNode armature_node = _scene.GetBoneNode(bone.Name);
                    Matrix4x4 bone_global_mat = armature_node.GlobTrans;
                    // bind tells the original delta in global coord, so we can find current delta
                    Matrix4x4 bind = bone.OffsetMatrix;
                    Matrix4x4 delta_roto = bind * bone_global_mat;
                    Matrix4x4 current_bone = delta_roto * current_node;
                    foreach (var pair in bone.VertexWeights)
                    {
                        // Can apply bone weight here
                        mesh_draw._vertex_id2matrix[pair.VertexID] = current_bone;
                    }
                }
            }
            foreach (Node child in nd.Children)
            {
                 RecursiveCalculateVertexTransform(child, current_node);
            }
        }

        /// <summary>Transform a Position by the given Matrix.
        /// Based on openTK compatiability vector 3 class.
        /// </summary>
        /// <param name="pos">The position to transform</param>
        /// <param name="mat">The desired transformation</param>
        /// <param name="result">The transformed position</param>
        public static void TransformPositionAssimp(ref Vector3D pos, ref Matrix4x4 mat, out Vector3D result)
        {
            // this is taken from https://github.com/opentk/opentk/blob/32665ca1cbdccb1c3be109ed0b7ff3f7cb5cb5b7/Source/Compatibility/Math/Vector3.cs
            // Note that assimp is row major, while opentk is column major
            result.X = pos.X * mat.A1 +
                       pos.Y * mat.A2 +
                       pos.Z * mat.A3 +
                       mat.A4;

            result.Y = pos.X * mat.B1 +
                       pos.Y * mat.B2 +
                       pos.Z * mat.B3 +
                       mat.B4;

            result.Z = pos.X * mat.C1 +
                       pos.Y * mat.C2 +
                       pos.Z * mat.C3 +
                       mat.C4;
        }

        // Second pass: transform all vertices in a mesh according to bone
        // just apply the previously caluclated matrix
        public void RecursiveTransformVertices(Node nd)
        {
            foreach (int mesh_id in nd.MeshIndices)
            {
                MeshDraw mesh_draw = _mesh_id2mesh_draw[mesh_id];
                // map data from VBO
                IntPtr data;
                int qty_vertices;
                mesh_draw.BeginModifyVertexData(out data, out qty_vertices);
                // iterate over inital vertex positions
                Mesh cur_mesh = _scene._inner.Meshes[mesh_id];
                MeshBounds aabb = _extra_geometry._mesh_id2box[mesh_id];
                // go over every vertex in the mesh
                unsafe
                {
                    // array of floats: X,Y,Z.....
                    int sz = 3; // size of step
                    float* coords = (float*)data;
                    for (int vertex_id = 0; vertex_id < qty_vertices; vertex_id++)
                    {
                        Matrix4x4 matrix_with_offset = mesh_draw._vertex_id2matrix[vertex_id];
                        // get the initial position of vertex when scene was loaded
                        Vector3D vertex_default = cur_mesh.Vertices[vertex_id];
                        Vector3D vertex;
                        Entity.TransformPositionAssimp(ref vertex_default, ref matrix_with_offset, out vertex);
                        // write new coords back into array
                        coords[vertex_id*sz + 0] = vertex.X;
                        coords[vertex_id*sz + 1] = vertex.Y;
                        coords[vertex_id*sz + 2] = vertex.Z;
                    }
                }
                mesh_draw.EndModifyVertexData();

                foreach (Node child in nd.Children)
                {
                    RecursiveTransformVertices(child);
                }
            }
        }

    } // end of class

}

\end{verbatim}
\subsection{Bone.cs}
\begin{verbatim}
using System;
using System.Collections.Generic;
using System.Linq;
using System.Text;
using System.Threading.Tasks;
using Assimp;
using Assimp.Configs;
using System.Windows.Forms;
using System.Drawing.Drawing2D;
using System.Drawing;
using System.IO;        // for MemoryStream
using System.Reflection;
using OpenTK;
using OpenTK.Graphics.OpenGL;
using System.Diagnostics;
using Quaternion = Assimp.Quaternion;

namespace WinFormAnimation2D
{
    struct BoundingVectors
    {
        public Vector3 ZeroNear;
        public Vector3 ZeroFar;
        public BoundingVectors(Vector3 near, Vector3 far)
        {
            ZeroNear = near;
            ZeroFar = far;
        }
    }

    class BoneBounds
    {
        public Vector3 _start;
        public Vector3 _end;

        // arbitrary vector that is perpendicular to the _end - _start
        // in 3D this might work better Vector3(-1*(_end.Y + _end.Z), 1, 1)
        // while in 2D use this Vector3(-1 * _end.Y, 1, 0), so that Z = 0;
        public Vector3 _normal
        {
            get {
                var bone_vec = _end - _start;
                var len = bone_vec.LengthFast;
                var sidevec = new Vector3(-1*(bone_vec.Y + bone_vec.X), 1.0f, 1.0f);
                return Vector3.Multiply(Vector3.NormalizeFast(sidevec), len/5.0f);
            }
        }

        public BoneBounds()
        {
            _start = Vector3.Zero;
            _end = Vector3.Zero;
        }

        public BoneBounds(Vector3 start, Vector3 end)
        {
            _start = start;
            _end = end;
        }

        // change from the 3d model into 2d program space just discard Z coordinate
        public Vector3[] Triangle
        {
            get
            {
                return new Vector3[] {
                    _start
                    , _start + _normal
                    , _end
                    , _start - _normal
                    , _start
                };
            }
        }

        public void Render(Pen p = null)
        {
            // Util.GR.DrawLines(p == null ? Pens.Aqua : p, tmp);
            GL.Enable(EnableCap.ColorMaterial);
            GL.Material(MaterialFace.FrontAndBack, MaterialParameter.AmbientAndDiffuse, Color.Aqua);
            GL.Color3(Color.Aqua);
            GL.LineWidth(3.0f);
            GL.Begin(BeginMode.LineLoop);
            foreach (Vector3 vec in Triangle)
            {
                GL.Vertex3(vec.X, vec.Y, vec.Z);
            }
            GL.End();
        }
    }

    /// Stores info on extra geometry of the entity, bones that is.
    class Geometry
    {

        public Dictionary<int,MeshBounds> _mesh_id2box = new Dictionary<int,MeshBounds>();
        /// Bone name matched up with the triangle to render.
        public Dictionary<string,BoneBounds> _bone_id2triangle = new Dictionary<string,BoneBounds>();
        public BoundingBoxGroup EntityBox;
        public double _average_bone_length;

        /// Build geometry data for node (usually use only for one of the children of scene.RootNode)
        public Geometry(IList<Mesh> scene_meshes, Node nd, BoneNode armature)
        {
            MakeBoundingBoxes(scene_meshes, nd);
            MakeBoundingTriangles(armature);
            _average_bone_length = FindAverageBoneLength(armature);
            UpdateBonePositions(armature);
            EntityBox = new BoundingBoxGroup(_mesh_id2box.Values);
        }
        


        /// For the length of final children bones. Just use average length.
        public double FindAverageBoneLength(BoneNode nd)
        {
            double len = 0;
            int qty = 0;
            InnerFindAverageLength(nd, ref len, ref qty);
            return len / qty;
        }

        public void InnerFindAverageLength(BoneNode nd, ref double total_length, ref int bones_count)
        {
            var triangle = _bone_id2triangle[nd._inner.Name];
            Vector3 bone_start = nd.GlobalTransform.ExtractTranslation();
            // dont analyse bones with no children
            if (nd.Children.Count > 0)
            {
                // this bone's end == the beginning of __any__ child bone
                Vector3 bone_end = nd.Children[0].GlobalTransform.ExtractTranslation();
                double len = (bone_start - bone_end).Length;
                total_length += len;
                bones_count++;
                foreach (var child_nd in nd.Children)
                {
                    InnerFindAverageLength(child_nd, ref total_length, ref bones_count);
                }
            }
        }

        /// Snap the render positions of bones, to deformations in the skeleton.
        public void UpdateBonePositions(BoneNode nd)
        {
            var triangle = _bone_id2triangle[nd._inner.Name];
            Vector3 new_start = nd.GlobalTransform.ExtractTranslation();
            if (nd.Children.Count > 0)
            {
                // this bone's end == the beginning of __any__ child bone
                Vector3 new_end = nd.Children[0].GlobalTransform.ExtractTranslation();
                triangle._start = new_start;
                triangle._end = new_end;
                foreach (var child_nd in nd.Children)
                {
                    UpdateBonePositions(child_nd);
                }
            }
            else
            {
                // this bone has no children, we don't know where it will end, so we guess.
                // strategy 1: just set a random sensible value for bone
                // strategy 2: get geometric center of the vertices that this bone acts on
                // we have to use the Y-unit vector instead of X because we defined Y_UP 
                // in the collada.dae file, so all the matrices work such that direct unit vector is unit Y
                // strategy 3: choose the length of the smallest bone found
                var delta = Vector3.TransformVector(Vector3.UnitY, nd.GlobalTransform);
                Vector3 new_end = new_start + Vector3.Multiply(delta, (float)_average_bone_length);
                triangle._start = new_start;
                triangle._end = new_end;
            }
        }

        // make triangles to draw for each bone
        private void MakeBoundingTriangles(BoneNode nd)
        {
            _bone_id2triangle[nd._inner.Name] = new BoneBounds();
            for (int i = 0; i < nd._inner.ChildCount; i++)
            {
                 MakeBoundingTriangles(nd.Children[i]);
            }
        }

        /// For each node calculate the bounding box.
        /// This is used to align the viewport nicely when the scene is imported.
        private void MakeBoundingBoxes(IList<Mesh> scene_meshes, Node node)
        {
            foreach (int index in node.MeshIndices)
            {
                Mesh mesh = scene_meshes[index];
                _mesh_id2box[index] = new MeshBounds();
            }
            for (int i = 0; i < node.ChildCount; i++)
            {
                MakeBoundingBoxes(scene_meshes, node.Children[i]);
            }
        }

        public MeshBounds IntersectWithMesh(Vector2 point)
        {
            foreach (MeshBounds border in _mesh_id2box.Values)
            {
                if (border.CheckContainsPoint(point))
                {
                    return border;
                }
            }
            return null;
        }

        public bool EntityBorderContainsPoint(Vector2 point)
        {
            return EntityBox.OverallBox.CheckContainsPoint(point);
        }

    }


}

\end{verbatim}
\subsection{MainForm.cs}
\begin{verbatim}
using Assimp;
using Assimp.Configs;
using System;
using System.Collections.Generic;
using System.ComponentModel;
using System.Data;
using System.Drawing;
using System.Drawing.Drawing2D;
using System.IO;
using System.Linq;
using System.Reflection;
using System.Text;
using System.Threading.Tasks;
using System.Windows.Forms;
using System.Diagnostics;
using System.Runtime.CompilerServices;
using OpenTK;
using OpenTK.Graphics.OpenGL;

namespace WinFormAnimation2D
{
    public partial class MainForm : Form
    {

        MouseState _mouse = new MouseState();

        private World _world;

        RecentFilesFolders Recent = new RecentFilesFolders();

        private Stopwatch _last_frame_sw = new Stopwatch();
        private double LastFrameDelay;

        private bool LoadOpenGLDone;

        // State of the camera currently. We can affect this with buttons.
        private GUIConfig _gui_conf = new GUIConfig();
        private CommandLine _cmd;

        private IHighlightableNode last_selected_node;

        private Entity _current;
        private Entity Current
        {
            get { return _world._enttity_one; }
            set {
                _current = value;
                _cmd._current = value;
            }
        }

        private int TrackBarTimeRange
        {
            get { return this.trackBar_time.Maximum - this.trackBar_time.Minimum; }
        }

        private KeyboardInput _kbd;

        // camera related stuff
        private CameraDevice _camera;

        public MainForm()
        {
            InitializeComponent();
            this.checkBox_OpenGLDrawAxis.Checked = Properties.Settings.Default.OpenGLDrawAxis;
            this.toolStripStatusLabel_AnimTime.Text = "";
            _kbd = new KeyboardInput();
            Matrix4 opengl_camera_init = Matrix4.LookAt(0, 50, 500, 0, 0, 0, 0, 1, 0).Inverted();
            _camera = new CameraDevice(opengl_camera_init);
            // manually register the mousewheel event handler.
            this.glControl1.MouseWheel += new MouseEventHandler(this.glControl1_MouseWheel);
            _world = new World();
            _cmd = new CommandLine(_world, this);
            Recent.CurrentlyOpenFilePathChanged
                += (new_filepath) => this.Text = "Current file: " + new_filepath;
            RefreshOpenRecentMenu();
        }

        /// <summary>
        /// Get the items to show in open recent menu
        /// </summary>
        private void RefreshOpenRecentMenu()
        {
            // just replace old menu item wth a new one to refresh it
            Recent.ReplaceOpenRecentMenu(this.recentToolStripMenuItem
                , filepath => OpenFileCollada(filepath)
            );
        }

        public void SetAnimTime(double val)
        {
            this.toolStripStatusLabel_AnimTime.Text = val.ToString("F4");
            // if the user is not working with the track bar
            if (! this.trackBar_time.Focused)
            {
                double factor = TrackBarTimeRange / Current._action.TotalDurationSeconds;
                int track_val = (int)(val * factor);
                this.trackBar_time.Value = track_val;
            }
        }

    }
}

\end{verbatim}
\end{scriptsize}
