\documentclass[
%a4paper,12pt
encoding=utf8
]{../twoeskd}

% \usepackage{eskdappsheet}

% Packages required by doxygen
\usepackage[export]{adjustbox} % also loads graphicx
\usepackage{graphicx}
\usepackage[utf8]{inputenc}
\usepackage{multicol}
\usepackage{multirow}
\usepackage{makeidx}

% NLS support packages
\usepackage[T2A]{fontenc}
\usepackage[russian]{babel}
\usepackage{pscyr}

% Font selection
\usepackage{courier}
\usepackage{amssymb}

\setlength{\parindent}{0cm}
\setlength{\parskip}{0.2cm}

% debug to see the frame borders
% from https://en.wikibooks.org/wiki/LaTeX/Page_Layout
% \usepackage{showframe}

% Indices & bibliography
\usepackage{natbib}
\usepackage[titles]{tocloft}
\setcounter{tocdepth}{3}
\setcounter{secnumdepth}{5}

% change style of titles in \section{}
\usepackage{titlesec}
\titleformat{\section}[hang]{\huge\bfseries\center}{\thetitle.}{1em}{}
\titleformat{\subsection}[hang]{\Large\normalfont\raggedright}{\thetitle.}{1em}{\underline}
\titleformat{\subsubsection}[hang]{\large\normalfont\raggedright}{\thetitle.}{1pt}{}

% Packages for text layout in normal mode
% \usepackage[parfill]{parskip} % автоматом делает пустые линии между параграфами, там где они есть в тексте
% \usepackage{indentfirst} % indent even in first paragraph
\usepackage{setspace}	 % controls space between lines
\setstretch{1} % space between lines
\setlength\parindent{0.9cm} % size of indent for every paragraph
\usepackage{csquotes}% превратить " " в красивые двойные кавычки
\MakeOuterQuote{"}


% this makes items spacing single-spaced in enumerations.
\newenvironment{my_enumerate}{
\begin{enumerate}
  \setlength{\itemsep}{1pt}
  \setlength{\parskip}{0pt}
  \setlength{\parsep}{0pt}}{\end{enumerate}
}


% Custom commands
% configure eskd
\titleTop{
\textbf{\Large ПРАВИТЕЛЬСТВО РОССИЙСКОЙ ФЕДЕРАЦИИ \\
НАЦИОНАЛЬНЫЙ ИССЛЕДОВАТЕЛЬСКИЙ УНИВЕРСИТЕТ \\
«ВЫСШАЯ ШКОЛА ЭКОНОМИКИ» } \\
\vspace*{0.2cm}
{\small Факультет компьютерных наук \\
Департамент программнoй инженерии \\
}
}
\titleDesignedBy{Студент группы БПИ 151 НИУ ВШЭ}{Абрамов А.M.}
\titleAgreedBy{%
\parbox[t]{7cm} {
Доцент департамента \\
программной инженерии \\
факультета компьютерных наук \\
канд. техн. наук \\
}}{Р. З. Ахметсафина}
\titleApprovedBy{
\parbox[t]{10cm} {
Академический руководитель \\
образовательной программы \\
«Программная инженерия» \\
профессор департамента программной \\
инженерии канд. техн. наук \\
}}{В. В. Шилов}
\titleName{ПРОГРАММА СКЕЛЕТНАЯ АНИМАЦИЯ}
\workTypeId{RU.17701729.509000 T3 01-1-ЛУ}

\titleSubname{Техническое задание}


%===== C O N T E N T S =====


\begin{document}

% Titlepage & ToC
\pagenumbering{roman}

% some water filling text, that is pointless but adds text
% \newpage

\begin{center}
{\large АННОТАЦИЯ}
\end{center}

Настоящий документ представляет собой техническое задание для разработки приложения реализации алгоритма скелетной анимации. Данный документ составлен в соответствии с ГОСТ. В документе содержатся следующие разделы: «Введение», «Основания для разработки», «Назначение разработки», «Требования к программе», «Технико-экономические характеристики», «Стадии и этапы разработки», «Порядок контроля и приемки».

В разделе «Введение» содержится информация о намиеновании и краткой характеристике разрабатываемого приложения.

В разделе «Основания для разработки» содержится информация о документах, на основании которых ведется разработка настоящего приложения, а так же наименование темы разработки.

В разделе «Назначение разработки» содержится информация о функциональном и эксплуатационном назначении разрабатываемого прилоежния.

В разделе «Требования к программному изделию» содержится информация о требованиях к функциональным характеристикам, требованиях к надежности, условиях эксплуатации, требованиях к составу и параметрам технических средств, требования к информационной и программной совместимости.

В разделе «Требования к программной документации» содержится информация о требованиях, в соответствии с которыми должна выполняться разработка программной документации приложения.

В разделе «Технико-экономические показатели» содержится информация об ориентировочной экономической эффективности и ожидаемой годовой потребности, экономических преимуществах разработки по сравнению с лучшими зарубежными и отечественными аналогами.

В разделе «Стадии и этапы разработки» содержится информация о необходимых стадиях разработки, этапах и содержании работ. Присутствует информация о сроках разработки и исполнителях.

В разделе «Порядок контроля и приемки» содержится подробная информация о видах испытаний, которые будут применены к данному приложению, а так же общих требованиях к приемке работ.

Перед прочтением настоящего документа рекомендуется ознакомиться со списком терминов, для предотвращения непонятных моментов.



\newpage
\pagenumbering{arabic}
\tableofcontents

% --- add my custom headers ---
\newpage
\section{Введение}
\subsection{Наименование}
Наименование: «Программа скелетная анимация». Англ.: «Program of Skeletal Animation».


\subsection{Краткая характеристика}
    Цель работы - реализовать систему скелетной анимации. 
    В основные задачи работы входит загрузка анимации из этих файлов, рассчет промежуточных кадров анимации и воспроизведение анимации на экране средствами OpenGL. 
    Также программа предоставляет пользовтелю возможность рассмотреть анимацию с разных ракурсов, перейти к любому моменту времени в анимации и просмотреть иерархию костей и модели.
В состав работы также входит создание демонстрационных исходных данных (файлов) для данной системы.



\newpage
\section{Основания для разработки}
\subsection{Документ, на основании которого ведется разработка}
Разработка программы ведется на основании приказа Национального исследовательского университета «Высшая школа экономики» \textnumero  6.18.1-02/1912-10 от 11.12.2015


\subsection{Наименование темы разработки}
Наименование темы разработки - «Программа скелетная анимация». \\
Разработка программы ведется в соответствии с учебным планом подготовки бакалавров по направлению «Программная инженерия», факультета Компьютерных наук.

\newpage
\section{Назначение разработки}
\subsection{Функциональное значение}
Функциональным назначением приложения является предоставление пользователю возможности быстро загрузить несколько анимаций из файла или нескольких файлов, просмотреть из, просмотреть информацию об отдельных составляющих каждой анимации, проиграть ее с разной скоростью и в произвольном напралении, проверить каждую анимацию на правильность формата.

\subsection{Эскплутационное значение}
Программа наглядно демонстрирует содержание файла экпортированного из пакетов для 3-х мерного моделированния. Она должна использоваться в процессе отладки приложений использующих анимацию и в работе дизайнера  3D моделей.


\newpage
\section{Требования к программному изделию}


%=========================================
\subsection{Требования к функциональным характеристикам}
\subsubsection{Состав выполняемых функций}
\subsubsection{Организация входных и выходных данных}
\subsubsection{Прочие требования}


%=========================================
\subsection{Требования к временным характеристикам}
Требования к временным характеристикам программы не предъявляются.


%=========================================
\subsection{Требования к интерфейсу}
Интерфейс должен соответствовать схеме интерфейса, указанной в приложении.


%=========================================
\subsection{Требования к надежности}
\subsubsection{Обеспечение устойчивого функционирования программы}
Программа не должна вне зависимости от входных данных или действий оператора завершатся аварийно. При некорректно введенных параметрах пользователю должно отображаться сообщение об ошибке внутри окна ввода около поля (или группы полей), в которое(-ые) было введено некорректное значение.
\subsubsection{Время восстановления после отказа}
Требования к восстановлению после отказа не предъявляются.
\subsubsection{Отказы из-за некорректных действий оператора}
При попытке запуска алгоритма при не всех введенных данных или данных введенных некорректно, пользователю должно выдаваться сообщение в окне MessageBox.


%=========================================
\subsection{Требования к условиям эксплуатации}
\subsubsection{Климатические условия}
Условия соответствующие условиям эксплуатации современных компьютеров.
\subsubsection{Вид обслуживания}
Приложение не требует каких-либо видов обслуживания.
\subsubsection{Численность и квалификация персонала}
Минимальное количество персонала, требуемого для работы программы: 1 оператор. Пользователь программы должен знать следующие понятия из линейной алгебры, программирования и 3-х мерного моделирования: вектор, кватернион, матрица направляющих косинусов, меш (англ. mesh), коренная вершина (англ. root node).


%=========================================
\subsection{Требования к составу и параметрам технических средств}
Для оптимальной работы приложения необходимо учесть следующие системные требования:
\begin{my_enumerate}
\item Компьютер, оснащенный:
    \begin{my_enumerate}
    \item 32-разрядный (x86) или 64-разрядный (x64) процессор с тактовой частотой 1 гигагерц (ГГц) или выше;
    \item 512 мегабайт (ГБ) оперативной памяти (ОЗУ);
    \item 1 гигабайт (ГБ) (для 32-разрядной системы) или 2 ГБ (для 64-разрядной системы) пространства на жестком диске;
    \item графическое устройство OpenGL с драйвером версии 3.1 или выше.
    \item видеоадаптер super VGA с расширением 800*600 либо более высоким.
    \end{my_enumerate}
\item Монитор
\item Видеокарта
\item Мышь
\item Клавиатура
\end{my_enumerate}


%=========================================
\subsection{Требования к информационной и програмной совместимости}
Исходный код программы обязательно должен быть написан с использованием языка C\#. Приложению необходим компьютер с поддержкой OpenGL версии не менее 3.1. Операционная система Windows 7 или более поздняя версия Windows. Должен быть установлен .NET Framework версии не ниже 2.0.



\newpage
\section{Требования к программной документации}
\subsection{Предварительный состав программной документации}
В обязательном порядке должны входить:
\begin{my_enumerate}
\item Техническое задание  (ГОСТ 19.201-78)
\item Пояснительная записка  (ГОСТ 19.404-79)
\item Руководство оператора  (ГОСТ 19.505-79)
\item Программа и методика испытаний (ГОСТ 19.301-79*)
\item Текст программы  (ГОСТ 19.401-78*)
\end{my_enumerate}



\newpage
\section{Технико-экономические показатели}
\subsection{Ориентировачная экономическая эффективность и годовая потребность}
Ориентировочная экономическая эффективность не рассчитывается, предполагается, что программа будет использоваться пользователем несколько раз в неделю, на протяжении коротких периодов времени, т. е. количество сеансов на одном рабочем месте в год составит примерно 48 сеансов.

\subsection{Экономические преимущества разработки}
Экономические преимущества разработки в сравнении с лучшими отечественными и зарубежными аналогами рассчитаны на январь 2016 года. Существующими аналогами данного приложения являются пакеты для 3-х мерного моделлирования и анимации. В силу того что данное приложение распростроняется бесплатно, единственным экономически выгодным аналогом к нему будет программа Blender. Однако Blender гораздо более сложен в использовании и потребляет намного больше системных ресурсов (жесткой памяти, ОЗУ, времени процессора).

\newpage
\section{Стадии и этапы разработки}

%=========================================
\subsection{Необходимые стадии разработки}

\subsubsection{Стадия разработки технического задания:}
\begin{my_enumerate}
\item Этап обоснования необходимости разработки программы:
    \begin{my_enumerate}
    \item постановка задачи.
    \item сбор исходных материалов.
    \end{my_enumerate}
\item Этап разработки и утверждения технического задания:
    \begin{my_enumerate}
    \item определение требований к программе.
    \item определение стадий, этапов и сроков разработки программы и документации на нее.
    \item согласование и утверждение технического задания.
    \end{my_enumerate}
\end{my_enumerate}

\subsubsection{Стадия разработки технического проекта:}
\begin{my_enumerate}
\item Этап разработки технического проекта:
    \begin{my_enumerate}
    \item разработка структуры и архитектуры программы.
    \item окончательное определение конфигурации технических средств.
    \end{my_enumerate}
\item Этап утверждения технического проекта:
    \begin{my_enumerate}
    \item разработка плана мероприятий по разработке программы
    \item разработка пояснительной записки.
    \end{my_enumerate}
\end{my_enumerate}


\subsubsection{Стадия разработки рабочего проекта:}
\begin{my_enumerate}
\item Этап разработки программы:
    \begin{my_enumerate}
    \item непосредственное программирование и отладка программы.
    \end{my_enumerate}
\item Этап разработки программной документации:
    \begin{my_enumerate}
    \item разработка следующих программных документов в соответствии с требованиями: техническое задание, пояснительная записка, руководство оператора, программа и методика испытания, текст программы, все в соответствии с требованиями ГОСТ 19.101-77.
    \end{my_enumerate}
\item Этап испытания программы:    
    \begin{my_enumerate}
    \item разработка, согласование и утверждение программы и методики испытаний.
    \item испытания программы.
    \item защита презентации, сдача разработанной документации.
    \item корректировка программы и программной документации по результатам испытаний.
    \end{my_enumerate}
\end{my_enumerate}


%=========================================
\subsection{Сроки работ и исполнители}
% TODO change date to the real date
Приложение должно быть разработано к 1 июля 2016 года, студентом группы БПИ151 Абрамовым Артемом.


\newpage
\section{Порядок контроля и приемки}
\subsection{Виды испытаний}
Контроль и приемка разработки осуществляются в соответствии с разработанным исполнителем и согласованным с заказчиком документом «Программа скелетная анимация» Программа и методика испытаний по (ГОСТ 19.301-79*).

\subsection{Требования к приемке работы}
Акт приемки-сдачи программы между исполнителем и заказчиком в эксплуатацию происходит при полной работоспособности программы, при выполнении указанных в настоящем документе функций и требований, при наличии документации к программе, выполненной в соответствии с требованиями настоящего технического задания.

\newpage
\section{Приложение 1. Терминология}
\subsection{Терминология}
\begin{description}

\item[Корневая вершина (англ. root node)]  
Самый верхний узел дерева.

\item[Полигональная сетка (жарг. меш от англ. polygon mesh)]
Совокупность вершин, рёбер и граней, которые определяют форму многогранного объекта в трехмерной компьютерной графике и объёмном моделировании. Гранями являются треугольники.

\item[Дерево]
Связный ациклический граф. Связность означает наличие путей между любой парой вершин, ацикличность — отсутствие циклов и то, что между парами вершин имеется только по одному пути.

\item[Степень вершины]
Количество инцидентных ей (входящих/исходящих из нее) ребер.

\item[Интерполяция, интерполирование анимации]
Способ нахождения промежуточных значений состояния анимации по имеющемуся дискретному набору известных значений.

\item[Z-буферизация]
В компьютерной трёхмерной графике способ учёта удалённости элемента изображения. Представляет собой один из вариантов решения «проблемы видимости»

\item[Z-конфликт (англ. Z–fighting)]
Если два объекта имеют близкую Z-координату, иногда, в зависимости от точки обзора, показывается то один, то другой, то оба полосатым узором.

\item[OpenGL (Open Graphics Library)]
Спецификация, определяющая независимый от языка программирования платформонезависимый программный интерфейс для написания приложений, использующих двумерную и трёхмерную компьютерную графику. На платформе Windows конкурирует с Direct3D.

\item[Рендеринг (англ. rendering — «визуализация»)]
Термин в компьютерной графике, обозначающий процесс получения изображения по модели с помощью компьютерной программы.

\item[Текстура]
Растровое изображение, накладываемое на поверхность полигональной модели для придания ей цвета, окраски или иллюзии рельефа. Приблизительно использование текстур можно легко представить как рисунок на поверхности скульптурного изображения.

\end{description}



\newpage
\section{Приложение 2. Схема интерфейса программы}
\begin{figure}[h!]
    \centering
    \includegraphics[width=0.8\textwidth]{interface.png}
    \caption{Схема интерфейса}
\end{figure}


\newpage
\section{Приложение 3. Список используемой литературы}
\subsection{Список используемой литературы}
\begin{my_enumerate}
\item
OpenGL Superbible: Comprehensive Tutorial and Reference (7th Edition)
Graham Sellers (Author), Richard S Wright Jr. (Author), Nicholas Haemel (Author)
ISBN-13: 978-0672337475

\item
Порев В.Н. Компьютерная графика. – СПб.: БХВ-Петербург, 2002. – 432 с.: ил.

\item
ГОСТ 19.102-77 Стадии разработки. //Единая система программной документации. -М.: ИПК Издательство стандартов, 2001.

\item
ГОСТ 19.201-78 Техническое задание. Требования к содержанию и оформлению // Единая система программной документации. -М.:ИПК Издательство стандартов, 2001.

\item
ГОСТ 19.101-77 Виды программ и программных документов
//Единая система программной документации. -М.: ИПК Издательство стандартов, 2.: 001.

\item
ГОСТ 19.103-77 Обозначения программ и программных документов. //Единая система программной документации. -М.: ИПК Издательство стандартов, 2001.

\item
ГОСТ 19.104-78 Основные надписи //Единая система программной документации. -М.: ИПК Издательство стандартов, 2001.

\item 
ГОСТ 19.105-78 Общие требования к программным документам. //Единая система
программной документации. – М.: ИПК Издательство стандартов, 2001.

\end{my_enumerate}



% Index
\newpage
\eskdListOfChanges

% \phantomsection
% \addcontentsline{toc}{section}{Алфавитный указатель}
% \printindex

\end{document}
