\subsection{Формат Collada (.dae)}
Collada — это формат, разработанный для обмена  информацией между 3D приложениями.
Управляется некоммерческой организацией Khronos Group. Collada использует открытый стандарт XML для обмена форматами, которые в противном случае были бы несовместимы. Collada был задуман как промежуточный формат для переноса файлов.
Реализована поддержка в таких программах, как Maya, 3ds Max, Blender.
Ниже приведен пример описания квадрата в формате collada:
\begin{scriptsize}
\begin{verbatim}
<?xml version="1.0" encoding="utf-8"?>
<COLLADA xmlns="http://www.collada.org/2005/11/COLLADASchema" version="1.4.1">
  <asset>
    <unit name="meter" meter="1"/>
    <up_axis>X_UP</up_axis>
  </asset>
  <library_images/>
  <library_geometries>
    <geometry id="Plane_001-mesh" name="Plane.001">
      <mesh>
        <source id="Plane_001-mesh-positions">
          <float_array id="Plane_001-mesh-positions-array" count="12">-1 -1 0 1 -1 0 -1 1 0 1 1 0</float_array>
          <technique_common>
            <accessor source="#Plane_001-mesh-positions-array" count="4" stride="3">
              <param name="X" type="float"/>
              <param name="Y" type="float"/>
              <param name="Z" type="float"/>
            </accessor>
          </technique_common>
        </source>
        <source id="Plane_001-mesh-normals">
          <float_array id="Plane_001-mesh-normals-array" count="3">0 0 1</float_array>
          <technique_common>
            <accessor source="#Plane_001-mesh-normals-array" count="1" stride="3">
              <param name="X" type="float"/>
              <param name="Y" type="float"/>
              <param name="Z" type="float"/>
            </accessor>
          </technique_common>
        </source>
        <polylist count="2">
          <input semantic="VERTEX" source="#Plane_001-mesh-vertices" offset="0"/>
          <input semantic="NORMAL" source="#Plane_001-mesh-normals" offset="1"/>
          <vcount>3 3 </vcount>
          <p>1 0 3 0 2 0 0 0 1 0 2 0</p>
        </polylist>
      </mesh>
    </geometry>
  </library_geometries>
  <library_visual_scenes>
    <visual_scene id="Scene" name="Scene">
      <node id="Plane_001" name="Plane_001" type="NODE">
        <matrix sid="transform">69.99999 0 0 0 0 5.28485e-6 -69.99999 0 0 69.99999 5.28485e-6 0 0 0 0 1</matrix>
        <instance_geometry url="#Plane_001-mesh" name="Plane_001"/>
      </node>
    </visual_scene>
  </library_visual_scenes>
  <scene>
    <instance_visual_scene url="#Scene"/>
  </scene>
</COLLADA>
\end{verbatim}
\end{scriptsize}
