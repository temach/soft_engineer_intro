\newpage

\begin{center}
{\large АННОТАЦИЯ}
\end{center}

Настоящий документ представляет собой техническое задание для разработки приложения реализации алгоритма скелетной анимации. Данный документ составлен в соответствии с ГОСТ. В документе содержатся следующие разделы: «Введение», «Основания для разработки», «Назначение разработки», «Требования к программе», «Технико-экономические характеристики», «Стадии и этапы разработки», «Порядок контроля и приемки».

В разделе «Введение» содержится информация о намиеновании и краткой характеристике разрабатываемого приложения.

В разделе «Основания для разработки» содержится информация о документах, на основании которых ведется разработка настоящего приложения, а так же наименование темы разработки.

В разделе «Назначение разработки» содержится информация о функциональном и эксплуатационном назначении разрабатываемого прилоежния.

В разделе «Требования к программному изделию» содержится информация о требованиях к функциональным характеристикам, требованиях к надежности, условиях эксплуатации, требованиях к составу и параметрам технических средств, требования к информационной и программной совместимости.

В разделе «Требования к программной документации» содержится информация о требованиях, в соответствии с которыми должна выполняться разработка программной документации приложения.

В разделе «Технико-экономические показатели» содержится информация об ориентировочной экономической эффективности и ожидаемой годовой потребности, экономических преимуществах разработки по сравнению с лучшими зарубежными и отечественными аналогами.

В разделе «Стадии и этапы разработки» содержится информация о необходимых стадиях разработки, этапах и содержании работ. Присутствует информация о сроках разработки и исполнителях.

В разделе «Порядок контроля и приемки» содержится подробная информация о видах испытаний, которые будут применены к данному приложению, а так же общих требованиях к приемке работ.

Перед прочтением настоящего документа рекомендуется ознакомиться со списком терминов, для предотвращения непонятных моментов.

