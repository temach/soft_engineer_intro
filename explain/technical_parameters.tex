\subsection{Постановка задачи на разработку программы}
Разработанный програмный продукт обязан:
\begin{my_enumerate}
\item Выполнять загрузку данных из специально отформатированного файла коллада (collada или .dae).
\item Проигрывать записанную в данном файле анимацию.
\item Предоставлять пользователю озможность перейти к любому моменту времени в анимации.
\item Отрисовывать кости модели.
\item Иметь возможность вкл./выкл. учититывание нормалей и материалов во время отрисовки.
\item Поддерживать два вида камер в OpenGL, первый вид это камера движение которой сковано орбитой вокруг модели и другой тип это камера двигающаяся совершенно свободно.
\end{my_enumerate}


\subsection{Описание алгоритма и функционирования программы}

\subsubsection{Изменение местоположения точек в 3-х мерном пространстве}




Алгоритм скелетной анимации основан на налиичии следующей информации:
\begin{my_enumerate}
\item Простого меша, который называется скелетом.
\item Позиции этого меша в определенные моменты времени.
\item Комплексного меша модели.
\item Связь каждой вершиной модели с "костью" из скелета.
\end{my_enumerate}

Для реализации алгоритма был написан класс применяющий к костям скелета деформацию соответствующую заданному моменту во времени. Деформация - это матрица переводящая старую пару координат, которые определяют кость, в новую пару. Однако она нам не известна. Так как позиции костей в скелете известны нам только в ключевые моменты времени, необходимо использовать алгоритмы интерполяции для нахождения матрицы деформации для данного кадра. 

\begin{lstlisting}
print "hello world from ApplyArmature.cs"
\end{lstlisting}

Каждая матрица деформации находиться относительно матризу деформации кости-родителя. Далее матрица деформации должна быть переведена из системы координат кости-родителя в систему координат меша скелета. Это позволяет применять их на вершинах в произвольном порядке, что необходимо для увеличения производительности. Ниже приведен код переводящий деформации кости в глобальные координаты:

\begin{lstlisting}
print "hello world from ApplyArmature.cs"
\end{lstlisting}


Каждый кадр необходимо для каждой вершине комплексного меша находить кости, которые имеют право на нее воздействовать и применять к вершине такую же деформацию, что была пременена и к костям. Ниже преведен код занимающийся поиском костей влияющих на вершину и применяющий к ней матрицу деформации каждой кости в зависимости от веса кости:

\begin{lstlisting}
print "hello world from Entity.cs"
\end{lstlisting}


\subsubsection{Работа с OpenGL}

Графический драйвер поддерживающий OpenGL, использует архитектуру наподобие клиент-сервер. Данные o вершинах, материалах и нормалях необходимо загружать в буферы памяти расположенные на видеокарте. Ниже приведен код для загрузки этих данных:

\begin{lstlisting}
print "hello world from DrawMesh.cs"
\end{lstlisting}

Для создания эффекта движения нобходимо менять каждый кадр менять содержимое буферов расположыенных на графической карте. А именно необходимо менять координаты вершин и направления нормалей к каждой вершине (для корректного отображения света/тени). Для этого необходимо послать запрос к драйверу OpenGL и получить указатель на память с загруженнуми данными. Так как эта память не принадлежит GCCollector, (обьекту .Net Framework отвечающему за контроллируемую память) то все манипуляции с ней должны производиться в блоке "unsafe". Далее приведен код модифицирущий данные в буфере для следующего кадра.

\begin{lstlisting}
print "hello world from DrawMesh.cs and Entity.cs"
\end{lstlisting}


\subsection{Обоснование выбора алгоритма решения задачи}
Выбор алгоритма скелетной анимации продиктован постановкой задачи, а именно пунктом о загрузке данных из специально отформатированного файла коллада (collada или .dae).

Перевод матриц деформации из системы координат кости-родителя в систему координат меша скелета позволяет применять их на вершинах в произвольном порядке, что необходимо для увеличения производительности. 

Данные o вершинах, материалах и нормалях необходимо загружать в буферы памяти расположенные на видеокарте для того что бы обеспечить приложению скорость не менее чем в 20 кадров в секунду для моделей остоящих более чем из $1,300,000$ треугольников.

Модифицировать буфер памяти необходимо после каждого кадра, в OpenGL существует специальный тип буферов для таких случаев создаваемый с помощю флага STREAM\_WRTIE. Далее для удобства и скорости создается словарь сопоставляющий индексу вершины, влияющие на нее кости и их камуллятивную матрицу деформации, это позволяет сэкономить на поиске подходящей кости в скелете.

\subsection{Mетод органзации входных и выходных данных}

\subsubsection{Описание метода входных и выходных данных}
Входными данными является файл в формате коллада, в котором в обязательном порядке должны присутствовать следующие элементы:
\begin{my_enumerate}
\item Oдин меш с соответствующим ему тегом <Name="Mesh">
\item Один трэк анимации.
\item Один скелет связанный с мешем, и с соответствующим ему тегом <Name="Armature">
\end{my_enumerate}

Выходными данными является отображение анимации на экране.

\subsubsection{Обоснование выбора метода  организации входных и выходных данных}
Если не выполненны условия на наличие меша, трэка анимации и связанного с мешем скелета то у программы не хватит информации для воспроизведения анимации. 
Наименования объектов, экспортируемых из 3-х мерного пакета для моделлирования, необходимы для того чтобы различать меш и скелет для меша, так как при экспорте они получают одинаковую структуру внутри файла.

\subsection{Выбор состава технических средств}

\subsubsection{Состав технических и програмных средств}
Для возможности запустить приложение необходимо учесть следующие системные требования:
\begin{my_enumerate}
\item Компьютер, оснащенный:
    \begin{my_enumerate}
    \item Обязательно 64-разрядный (x64) процессор с тактовой частотой 1 гигагерц (ГГц) или выше;
    \item 512 мегабайт (ГБ) оперативной памяти (ОЗУ);
    \item 2 ГБ (для 64-разрядной системы) пространства на жестком диске;
    \item графическое устройство OpenGL с драйвером версии 3.1 или выше.
    \end{my_enumerate}
\item Монитор
\item Видеокарта
\item Мышь
\item Клавиатура
\end{my_enumerate}

\bigskip

Также необходимо учесть следующие програмные требования:
\begin{my_enumerate}
\item Поддержка OpenGL версии не менее 3.1
\item 64-битная операционная система Windows 7.
\item .NET Framework версии не ниже 4.5.1
\item Библиотека Assimp версии не ниже 3.1 
\item Библиотека OpenTK версии не ниже 1.1.4
\end{my_enumerate}

\subsection{Обоснование выбора технических и програмных средств}
Программа была протестирована и отлажена на версии OS Windows 7 с использованием .Net Framework 4.5.1, OpenTK версии 1.1.4 и Assimp версии 3.1.

Качество и корректность работы программы при других версиях не проверялось.

Программа использует буферы графической памяти типа STREAM\_WRITE и функции glMapData и glSubBufferData которые в OpenGL официально поддерживаются лишь с версии 3.1

64-x разрядный компьютер необходим для того чтобы упростит логику взаимодействия в блоках кода unsafe с внешней паматью выделяемой графическим драйвером.

Технические требования к памяти и периферии не превышают технических требований к операционной системе Windows 7 с установленным на ней .Net Framework 4.5.1
