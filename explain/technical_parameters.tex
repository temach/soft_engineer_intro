\subsection{Постановка задачи на разработку программы}
    Цель работы - реализовать программу, обеспечивающую просмотр анимации из файла предназначенного для неявных систем скелетной анимации.
    
\bigskip

Основные задачи работы:

\smallskip
\begin{my_enumerate}
\item Загрузка анимации из файла (содержание описанно в Т.З).
\item Рассчет кадров анимации.
\item Воспроизведение анимации на экране средствами OpenGL.
\end{my_enumerate}
	
Второстепенные задачи работы:
\begin{my_enumerate}
\item Предоставлять пользователю озможность перейти к любому моменту времени в анимации.
\item Отрисовывать кости модели.
\item Иметь возможность вкл./выкл. учититывание нормалей и материалов во время отрисовки.
\item Поддерживать два вида камер в OpenGL, первый вид это камера движение которой сковано орбитой вокруг модели и другой тип это камера двигающаяся совершенно свободно.
\end{my_enumerate}


\subsection{Описание алгоритма и функционирования программы}


%=============================================================
\subsubsection{Выбор алгоритма}

\paragraph{Различные подходы}
к созданию систем 3-х мерной анимации балансируют между методами с большим количеством вычислений и методами требующими большого объема памяти. Условно можно выделить явные и неявные системы анимации.

\paragraph{Явные системы анимации} хранят отдельную модель для каждого кадра.
После записи, существует много методов для воспроизведения анимации.
Такие методы требуют лишь элементарной математики.
Однако типичная запись одного трэка анимации для одного персонажа занимает около 10MB (в формате MD3).

\begin{figure}[h!]
    \centering
    \includegraphics[width=0.8\textwidth]{explicit_animation.png}
    \caption{Каждому кадру соответствует своя модель}
\end{figure}

Предпочтение явным системам отдается когда необходимо анимировать большие группы людей или животных.

\paragraph{Неявные системы анимации} хранят не модели, а более высокоуровневое описание движения.
В частности системы скелетной анимации содержат описание (через матрицу поворота) для каждой кости, как например локоть, плечо, шея.
В реальном времени эти описания применяются к неанимированной модели для рассчета следующего кадра анимации.
Эти рассчеты обычно требуют сложной математики с матрицами и тригонометрией.
А следовательно и много CPU времени.


\begin{figure}[h!]
    \centering
    \includegraphics[width=0.8\textwidth]{implicit_animation.png}
    \caption{\scriptsize{Слева: анимированный персонаж; справа: скелет для данного кадра}}
    %\label{fig:awesome_image}
\end{figure}

Неявные системы используются когда действия персонажа связанны с другими предметами и нельзя предугадать все возможные варианты анимации. Например, для того чтобы ставить ступню параллельно поверхности при движении по неровной земле.


%=============================================================
\subsubsection{Описание алгоритма и структур данных}
Для реализации требуются 3 вещи: 
\begin{description}
\item[Скелет] Скелет определяет иерархию частей тела персонажа. 
Скелет, - это дерево из костей.

\item[Трэк анимации] В трэке содержатся матрицы поворота скелета в ключевые моменты времени.
Трэк, - это массив пар (время, набор матриц поворота).

\item[Модель]  - это полигональная модель.
\end{description}

Для создания эффекта анимации необходимо извлекать из трэка набор матриц поворота соответствующий настоящему моменту времени. Применять этот набор матриц к скелету, а затем применять позу скелета к модели.

\paragraph{Применение описания из трэка к скелету}
Матрицы поворота для всех костей записываются в трэке относительно матрицы поворота родителя.
Поэтому для деформации скелета необходимо применять матрицы последовательно.
Мы начинаем с корневой кости и применяем к ней описанную в трэке анимации матрицу поворота.
Затем, мы двигаемся вглубь скелета применяя матрицу родителя и матрицу описанную в трэке (то есть считаем глобальную матрицу поворота для данной кости), пока не рассмотрим все кости.

\begin{figure}[h!]
    \centering
    \includegraphics[width=0.8\textwidth]{forward_kinematics.png}
    \caption{\scriptsize{Применение преобразований, начиная от копчика (корневой кости) и заканчивая ступней.}}
    %\label{fig:awesome_image}
\end{figure}

\paragraph{Применение деформации скелета к модели}

После того как рассчитанны матрицы поворотов для скелета, их необходимо применить на вершины модели.
Для этого используется рекурсивный алгоритм очень похожий на предыдущий.

\begin{scriptsize}
\begin{verbatim}
deform (bone root, mesh original, mesh deformed)
  for each child_bone of root
    for each vertex in the original mesh
      if bone_weight > 0
          apply bone global transform to vertex
          scale the resulting point by the bone weight
          store the result in deformed
      end if
    end for
    if child_bone has children
      deform (children of this node, mesh original, deformed)
    end if
  end for
\end{verbatim}
\end{scriptsize}


\subsubsection{Реализация системы скелетной анимации}

Для реализации  было созданно несколько функциональных блоков.
Диаграмма основных блоков:

\begin{figure}[h!]
    \centering
    \includegraphics[width=0.8\textwidth]{block_diagram.png}
    \caption{\scriptsize{Схема логических блоков в программе}}
   %\label{fig:awesome_image}
\end{figure}


\paragraph{Описание системы, блок чтения файла}
Заполнение структур данными из файла.
С помощью библиотеки Assimp производится чтение из файла. Для оптимальной работы данные перераспределяются из структур Assimp в свои. Другие функции этой библиотеки не используются.
Ниже приведен упрощенный код считывающий информацию из файла и строящий стуктуры данных.

\begin{scriptsize}
\begin{verbatim}
public void LoadScene(byte[] filedata)
{
    using (MemoryStream fs = new MemoryStream(filedata))
    {
        _cur_scene = new SceneWrapper(ReadAssimpScene(fs, "dae"));
        // у входных данных всегда только один трэк анимации
        _animation_track = _cur_scene.Animations[0];
        // блок применения описаний из трэка анимации к скелету
        _action = new Animator(_animation_track);
        // корневая кость скелета
        BoneNode bones = _cur_scene.BuildBoneNodes("Armature");
        // модель
        Node mesh = _cur_scene.FindNode("Mesh");
        // блок состояния анимации
        ActionState state = new ActionState();
        // блок персонажа
        _enttity = new Entity(_cur_scene, mesh, bones, state);
    }
}
\end{verbatim}
\end{scriptsize}


\paragraph{Описание системы, блок состояния анимации}
Хранит состояние анимации. Наиболее важные поля:
\begin{itemize}
\item Название трэка анимации.
\item Настоящий момент времени в секундах.
\item Индексы всех ключевых кадров и время каждого ключевого кадра.
\end{itemize}

Есть фyнкция SetTime(\dots) для перехода к определенному моменту времени. Она находит интервал между ключевыми кадрами, подсчитывает величину интерполяции.

\begin{verbatim}
// функция в блоке состояния анимации для прыжка к определенному времени
public void SetTime(double time_seconds)
{            
    double time_ticks = time_seconds * TickPerSec;
    // когда время в секундах переполняеться, запускаем анимацию с начала
    double time = time_ticks % TotalDurationTicks;
    // поиск интервала между ключевыми кадрами
    int start_frame = FindStartFrameAtTime(time_seconds);
    int end_frame = (start_frame + 1) % KeyframeCount;
    // нахождение значения для интерполяции между ключевыми кадрами
    double delta_ticks = KeyframeTimes[end_frame] - KeyframeTimes[start_frame];
    // если начали анимацию заново
    if (delta_ticks < 0.0)
    {
        delta_ticks += TotalDurationTicks;
    }
    double blend = (time - KeyframeTimes[start_frame]) / delta_ticks;
    // приписываем результаты рассчетов
    OriginKeyframe = start_frame;
    TargetKeyframe = end_frame;
    KfBlend = blend;
}
\end{verbatim}



\paragraph{Описание системы, блок хранения данных}
Работает со скелетом и моделью.
Реализует функции поиска костей в скелете или подмоделей в модели.
Функция BuildBones строит скелет по данным из модели (скелет как отдельный класс не существует, он определяеться корневой костью).

Ниже приведен класс описывающий кость скелета:
\begin{verbatim}
class BoneNode
{
    public string Name;
    public Matrix4 GlobalTransform;
    public Matrix4 LocalTransform;

    public BoneNode Parent;
    public List<BoneNode> Children;
    public BoneNode(Node assimp_node) { ... }
}
\end{verbatim}


\paragraph{Описание системы, блок деформации скелета}
Применяет данные описывающие (в матрицах поворота) новую позицию для каждой кости к костям из скелета.
То есть деформирует скелет в соответствии с моментом времени в анимации. На вход блока подается класс ActionState содержащий информацию о состоянии анимации и корневая кость скелета.

\begin{verbatim}
// функция для извлечения матриц поворота из трэка и применения их к скелету
public void ChangeLocalFixedDataBlend(ActionState st)
{
    // для каждой кости создает свой канал анимации    
    foreach (NodeAnimationChannel channel in _action.NodeAnimationChannels)
    {
        BoneNode bone_nd = _scene.GetBoneNode(channel.NodeName);
        // поворот кости
        Quaternion target_roto = Quaternion.Identity;
        if (channel.RotationKeyCount > st.TargetKeyframe)
        {
            target_roto = channel.RotationKeys[st.TargetKeyframe].Value.eToOpenTK();
        }
        Quaternion start_frame_roto = channel.RotationKeys[st.OriginKeyframe].Value;
        // интерполяция поворота между двумя ключевыми кадрами
        Quaternion result_roto = Quaternion.Slerp(start_frame_roto, target_roto, (float)st.KfBlend);
        // сдвиг кости
        Vector3 target_trans = Vector3.Zero;
        if (channel.PositionKeyCount > st.TargetKeyframe)
        {
            target_trans = channel.PositionKeys[st.TargetKeyframe].Value;
        }
        Vector3 cur_trans = channel.PositionKeys[st.OriginKeyframe].Value;
        // интерполяция сдвига между двумя ключевыми кадрами
        Vector3 result_trans = cur_trans + Vector3.Multiply(target_trans - cur_trans, (float)st.KfBlend);
        // объединение поворота и сдвига
        Matrix4 result = Matrix4.CreateFromQuaternion(result_roto);
        result.Row3.Xyz = result_trans;
        bone_nd.LocalTransform = result;
    }
}

// функция для рассчета глобальной матрицы поворота для каждой кости
// эта матрица будет позднее применена к вершинам модели
private void ReCalculateGlobalTransform(BoneNode nd)
{
    nd.GlobalTransform = nd.LocalTransform * nd.Parent.GlobalTransform;
    foreach (var child in nd.Children)
    {
        ReCalculateGlobalTransform(child);
    }
}
\end{verbatim}



\paragraph{Описание системы, блоки отрисовки модели}
Загружает данные о модели в OpenGL.
Запрашивает OpenGL об отводе буферов памяти под вершины, нормали, цвета вершин и массив индексов. Применяет свойства материала, например: цвет, коэффициент рассеивания света, коэффициент свечения и т.д.

Данные o вершинах, материалах и нормалях необходимо загружать в буферы памяти расположенные на видеокарте для того что бы обеспечить приложению приемлимую скорость отрисовки.
Ниже приведен код для загрузки данных в память видеокарты:

\begin{verbatim}
// объект содержащий номера буферов в OpenGL
struct Vbo
{
    public int VertexBufferId;
    public int NormalBufferId;
    public int ElementBufferId;
    public int NumIndices;
}

// функция для создания нового буфера 
// и заполнения его данными из массива векторов
private void NewOpenGLBufferWithFloats(out int outGlBufferId, List<Vector3D> dataBuffer) 
{
    GL.GenBuffers(1, out outGlBufferId);
    GL.BindBuffer(BufferTarget.ArrayBuffer, outGlBufferId);
    int sizeof_vec3d = 12; // X,Y,Z = 3 floats, 4 bytes each
    var byteCount = dataBuffer.Count * sizeof_vec3d;
    var temp = new float[byteCount];
    var n = 0;
    foreach(var v in dataBuffer)
    {
        temp[n++] = v.X;
        temp[n++] = v.Y;
        temp[n++] = v.Z;
    }
    GL.BufferData(BufferTarget.ArrayBuffer, (IntPtr)byteCount, temp, BufferUsageHint.StreamDraw);
    VerifyArrayBufferSize(byteCount);
    GL.BindBuffer(BufferTarget.ArrayBuffer, 0);
}

// функция для загрузки данных о модели в память видеокарты
private void Upload(out Vbo vboToFill)
{
    vboToFill = new Vbo();    
    NewOpenGLBufferWithFloats(out vboToFill.VertexBufferId, _mesh.Vertices);
    if (_mesh.HasNormals)
    {
        NewOpenGLBufferWithFloats(out vboToFill.NormalBufferId, _mesh.Normals);
    }
}

\end{verbatim}

Для создания эффекта движения нобходимо каждый кадр менять содержимое буферов расположыенных на видеокарте.
А именно необходимо менять координаты вершин и направления нормалей к каждой вершине (для корректного отображения света/тени).
Для этого необходимо послать запрос к драйверу OpenGL и получить указатель на память с загруженными ранее данными.
Далее приведен код модифицирущий данные в буфере для следующего кадра.

\begin{verbatim}
// функция в блоке отрисовки модели для получения доступа к буферу OpenGL
public void BeginModifyNormalData(out IntPtr data, out int qty_normals)
{
    GL.BindBuffer(BufferTarget.ArrayBuffer, _vbo.NormalBufferId);
    data = GL.MapBuffer(BufferTarget.ArrayBuffer, BufferAccess.ReadWrite);
    qty_normals = _mesh.Normals.Count;
}
// функция в блоке отрисовки модели для освобождения буфера OpenGL
public void EndModifyNormalData()
{
    bool data_upload_ok = GL.UnmapBuffer(BufferTarget.ArrayBuffer);
    if (! data_upload_ok)
    {
        throw new Exception("OpenGL driver has failed.");
    }
}

// функция в блоке персонажа для применения матриц костей к вершинам модели
public void RecursiveTransformVertices(Node node)
{
    foreach (int mesh in nd.Meshes)
    {
        // получаем указатель на буфер в OpenGL
        IntPtr pbuf_opengl;
        int qty_vertices;
        mesh.BeginModifyVertexData(out pbuf_opengl, out qty_vertices);
        // изначальная модель без деформаций
        Mesh original_data = mesh.OriginalData;
        // go over every vertex in the mesh
        unsafe
        {
            int sz = 3;         // размер шага
            float* coords = (float*)pbuf_opengl;
            for (int vertex_id = 0; vertex_id < qty_vertices; vertex_id++)
            {
                Matrix4 matrix_with_offset = mesh._vertex_id2matrix[vertex_id];
                // получить изначальные координаты вершины
                Vector3 vertex_default = original_data.Vertices[vertex_id];
                Vector3 vertex;
                Vector3.Transform(ref vertex_default, ref matrix_with_offset, out vertex);
                // Применение веса к вершине
                Vector3 delta = vertex_default - vertex;
                vertex += delta *  mesh._vertex_id2bone_weight[vertex_id];
                // Запись новых координат обратно в буфер OpenGL
                coords[vertex_id*sz + 0] = vertex.X;
                coords[vertex_id*sz + 1] = vertex.Y;
                coords[vertex_id*sz + 2] = vertex.Z;
            }
        }
        mesh.EndModifyVertexData();

        foreach (Node child in nd.Children)
        {
            RecursiveTransformVertices(child);
        }
    }
}
\end{verbatim}


\paragraph{Описание системы, блок прерсонажа}
Объединяет компоненты необходимые для анимации одного персонажа. Хранит ссылки на скелет (корневую кость), состояние анимации (ActionState), на саму модель и на класс отрисовки модели (MeshDraw).
    \medskip
    В частности блок персонажа применяет трансформации из скелета к вершинам модели (взвешивая действие каждой кости на вершину) и модифицирует данные в буфере данных OpenGL, что и создает эффект анимации.


\subsection{Mетод органзации входных и выходных данных}

\subsubsection{Описание метода входных и выходных данных}
Входными данными является файл в формате collada (.dae)
, в котором в обязательном порядке должны присутствовать следующие элементы:
\begin{my_enumerate}
\item Одна полигональная модель.
\item Один трэк анимации.
\item Один скелет.
\end{my_enumerate}

Если не выполненны условия на наличие полигональной модели,
трэка анимации и связанного с моделью скелета то
у программы не хватит информации для воспроизведения анимации.

Выходными данными является отображение анимации на экране.


\subsection{Выбор состава технических средств}

\subsubsection{Состав технических и програмных средств}
Для возможности запустить приложение необходимо учесть следующие системные требования:
\begin{my_enumerate}
\item Компьютер, оснащенный:
    \begin{my_enumerate}
    \item Обязательно 64-разрядный (x64) процессор с тактовой частотой 1 гигагерц (ГГц) или выше;
    \item 512 мегабайт (ГБ) оперативной памяти (ОЗУ);
    \item 2 ГБ (для 64-разрядной системы) пространства на жестком диске;
    \item графическое устройство OpenGL с драйвером версии 3.1 или выше.
    \end{my_enumerate}
\item Монитор
\item Видеокарта
\item Мышь
\item Клавиатура
\end{my_enumerate}

\bigskip

Также необходимо учесть следующие програмные требования:
\begin{my_enumerate}
\item Поддержка OpenGL версии 3.1
\item 64-битная операционная система Windows 7.
\item .NET Framework версии 4.5.1
\item Библиотека Assimp версии 3.1
\item Библиотека OpenTK версии 1.1.4
\end{my_enumerate}

Программа была протестирована и отлажена на версии OS Windows 7 с использованием .Net Framework 4.5.1, OpenTK версии 1.1.4 и Assimp версии 3.1.

Качество и корректность работы программы при других версиях библиотек и операционных систем не проверялось.

Программа использует буферы графической памяти типа STREAM\_WRITE и функции glMapData и glSubBufferData которые в OpenGL официально поддерживаются лишь с версии 3.1

Технические требования к памяти и периферии не превышают технических требований к операционной системе Windows 7 с установленным на ней .Net Framework 4.5.1
