\documentclass[a4paper,12pt]{report} % формат бумаги А4, шрифт по умолчанию - 12pt

% заметь, что в квадратных скобках вводятся необязательные аргументы пакетов.
% а в фигурных - обязательные

\usepackage[T2A]{fontenc} % поддержка кириллицы в Латехе
\usepackage[utf8]{inputenc} % включаю кодировку ютф8
\usepackage[english,russian]{babel} % использую русский и английский языки с переносами

\usepackage{multicol} % подключаю мультиколоночность в тексте, которая обычно нафиг не нужна
\usepackage{graphicx} % пакет для вставки графики, я хз нахуя он нужен в этом документе
\usepackage{listings} % пакет для вставки кода

\usepackage{amsmath} % математические штуковины
\usepackage{mathtools} % еще математические штуковины
\usepackage{mathtext}
\usepackage{verbatim} % пакет для коммнтарий и вставки текста напрямую
\usepackage[parfill]{parskip} % автоматом делает пустые линии между параграфами, там где они есть в тексте

\usepackage{setspace}	 % контроль за размером между строками
\setstretch{1} % ширина между строками 1.2

\usepackage{geometry} % меняю поля страницы
% из параметров ниже понятно, какие части полей страницы меняются:
\geometry{left=2.5cm}
\geometry{right=2cm}
\geometry{top=2cm}
\geometry{bottom=2cm}

\setcounter{secnumdepth}{0} % suppress heading/section/etc numberings

\righthyphenmin=2

\usepackage{indentfirst} % делать отступ в самом первом параграфе, хотя и так ясно что это новый параграф
\setlength\parindent{0pt}	% размер отступа в начале каждого параграфа

\usepackage{csquotes}% превратить " " в красивые двойные кавычки
\MakeOuterQuote{"}		

\begin{document}

\begin{titlepage}

\newpage

\begin{center}

{\large 
НАЦИОНАЛЬНЫЙ ИССЛЕДОВАТЕЛЬСКИЙ УНИВЕРСИТЕТ \\
«ВЫСШАЯ ШКОЛА ЭКОНОМИКИ» 	\\
Дисциплина: «Психология» 	\\
}

\vfill % заполняет длину страницы вертикально

{\large 
	Домашнее задание \\
	Анализ фильма \underline{«Гарольд и Мод (Harold and Maude)»} 	\\
}

\bigskip

\vfill

\begin{flushright}
Выполнил: Абрамов Артем,\\
студент группы БПИ151 \\
%\medskip
факультета компьютерных наук \\
отделения программной инженерии \\
\end{flushright}

\vfill

Москва \number\year

\end{center}
\end{titlepage}

\newpage

%\begin{flushleft}

\begin{center}
{\normalsize
 	Гарольд и Мод (Harold and Maude) \\
 	1971 г. режиссер Хэл Эшби \\
}
\end{center}

\subsubsection{Введение}

Этот фильм произвел на меня сильное впечатление, когда я смотрел его впервые. Такую кинокартину можно пересматривать по множеству раз, постоянно находя в ней что-то новое. Мне показалось интересным вновь вспомнить героев, переосмыслить сюжет, и проанализировать встречающиеся в нем психологические феномены.

Это довольно старая лента (70 годов), тем не менее она не потеряла актуальности и сегодня. Конечно не все из нас как главный герой фильма живут в особняке, страдают навязчивыми мыслями о смерти и водят катафалк. Однако это просто комедийный фон, созданный для иллюстрации гораздо более глубоких и нравственных вопросов о жизни. 

При первом просмотре на меня произвел большое впечатление рекламный плакат (представлен ниже). Главный герой очевидно выглядит неординарно, держа в руках нож, пистолет и канистру с бензином.

Фильм запомнился мне тем, что в нем поднимались вопросы взросления, автономности, независимости и, конечно, жизни и смерти. Естественно это вечные вопросы и киноленты, рассматривающие их, еще не скоро устареют, точно так же, как и некоторые книги могут сохранять актуальность на протяжении многих лет.

Ну и наконец в нем есть как откровенно комедийные, так и философские моменты, позволяющие пересматривать его по нескольку раз, постоянно открывая для себя новые нюансы и тонкости киноискусства.

Как уже говорилось, в фильме "Гарольд и Мод" можно встретить достаточно много психологических феноменов, я выделил те из них, которые показались мне наиболее интересными. В качестве источников для уточения определений и терминов мной использовались книги:  "Психология" авторы Нуркова В.В., Березанская Н.Б., "Большой психологический словарь" авторы Б. Г. Мещеряков, В. П. Зинченко, "Общая психология" автор  А. Г. Маклаков.


\begin{figure}[!h]
	\centering
	\includegraphics[scale=0.25]{Harold_and_Maude_poster}
	\caption{Постер фильма}
\end{figure}

\newpage


\subsubsection{\underline{Взросление, автономия}}

Под влиянием матери Гарольд растет человеком не интересующимся жизнью и далеко не самостоятельным. Однако знакомство с Мод начинает менять его пассивный взгляд на жизнь. Впервые столкнувшись с ее нестандартным поведением он решил, что она сумасшедшая, но это его скорее заинтриговало. Постепенно, наблюдая за тем, как она снова и снова идет против общепринятых норм в своей неугасающей борьбе за то, что она называет "великими ценностями", он учиться критически мыслить.  В терминах возрастной психологии он переходит к стадии нравственного развития называемой автономией.

Окидывая свою жизнь и людей, и окружение вокруг себя новым взглядом, Гарольд понимает, что те законы, которые навязывает ему общество и которыми он откровенно тяготиться, не являются истиной в последней инстанции. С детства родители вбивали в него правила этикета и приучали его действовать по их логике. Теперь он в состоянии сознательно отказаться от всего этого. Например, при первой поездке в дом Мод, он вскоре вынужден покинуть ее, поскольку должен попасть на сеанс с психиатором. Он хочет провести время с Мод, но все же повинуется желанию матери и едет на консультацию. Ближе к концу фильма мы видим как Гарольд со спокойной совестью пропускает очередной сеанс поскольку уже четко понимает, что ему это неинтересно да и не нужно. Более того, поняв что он любит Мод, он без тени сомнения и совершенно открыто заявляет о том, что нашел свою будущую невесту, хотя прекрасно понимает, что его мать будет ошеломлена и недовольна таким выбором. Из этого видно, что Гарольд взрослеет и начинает формировать свои собственные принципы и моральные ценности, а не слепо следовать чужой указке. 

Переосмысление своего взгляда на жизнь обычно происходит в подростковом возрасте. Поэтому период автономии часто рассматривают как третий этап развития личности, после гетерономии (или конвенциональной морали) и аномии (доморального этапа). С моей точки зрения общение с другими людьми (которого Гарольд был лишен) играет ключевую роль в переходе к автономии. Именно понимание разнообразных подходов к жизненным проблеммам позволяет человеку осознанно выбирать свой путь в мире и мыслить самому за себя.


\subsubsection{\underline{Авторитарность}}

Миссис Чэйзен несомненно авторитарная личность. Ее приоритетная задача - контроль над развитием Гарольда. Например, за ужином с гостями она забирает в свои руки инициативу и принимается рассказывать про жизнь Гарольда, хотя ему уже больше 20 лет, и он несомненно мог бы сам принять участие в разговоре, если бы она дала ему шанс. Более того, примерно в середине фильма  именно Миссис Чейзен решает, что Гарольду пора бы уже жениться. Считая его не в состоянии подобрать себе подходящую жену, она берется записывать его в национальную службу знакомств, но как она это делает! Полностью увлекшись процессом бланков нового участника она совершенно забывает о том, что вопросник должен отражать мнение сына, а не её собственное. Такого рода ситуации постоянно возникают когда один человек считающий себя наиболее опытным, умным и образованным решает "помочь" другому. Фактически он берет решение проблеммы в свои руки, отнимая у других людей, какой-либо шанс на инициативу и саморазвитие, таким образом он ставит их в зависимость от себя.


\begin{figure}[!h]
	\centering
	\includegraphics[scale=0.4]{fill_the_form}
	\caption{Миссис Чэйзен отвечает на вопросы вместо Гарольда}
\end{figure}

\newpage

Приведу часть монолога Миссис Чэйзен: \\

{\small
 - Эту анкету нужно заполнить и вернуть. Ты готов, Гарольд? \\
 - Вот первый вопрос, Гарольд. "Трудно ли вам общаться с незнакомыми людьми?" Я думаю, ответ ДА. Ты же согласен, Гарольд? \\
 - "Нужно ли вводить в школе уроки полового воспитания?" О, я бы сказала НЕТ. Ну поставим два балла. \\
 - "Может ли женщина стать президентом США?" Почему же нет. Несомненно может. \\
 - "Запоминаете ли вы анекдоты и любите ли их рассказывать?" Ты ведь этого не делаешь, Гарольд. НЕТ. Несомненно НЕТ. \\
 - "Бывает ли у вас чувство что жизнь потеряла смысл?" Как думаешь, Гарольд - 5, 4? О, поставим 3 - НЕ УВЕРЕН. У-хмм. \\
}

\subsubsection{\underline{Рыночный характер (по Эриху Фромму)}}

Миссис Чэйзен - это пример богатой и зацикленной на обществе женщины. Ее характер проявляется в постоянном желании произвести на гостей впечатление своим домом, платьем и даже сыном, пусть даже он снова и снова выводит ее из себя. Она хочет чтобы у него было все самое лучшее с материальной точки зрения, уделяя слишком мало внимания духовным ценностям. Здесь проявляется то, что с точки зрения Эриха Фромма можно охарактеризовать рыночным характером. Фромм дает развернутую критику западного общества, которое зиждется на принципе неограниченного потребления как цели жизни. Это отражено в поведении матери, например, она покупает Гарольду новую и модную машину (Jaguar), и выбрасывает тот катафалк на котором он ездил. Хотя катафалк, конечно, не самый практичный вид автотраспорта, но все же это был выбор именно Гаральда. Миссис Чейзен с легкостью отмела мнение сына как глупое чудачество, поскольку согласно своему рыночному менталитету не могла понять как кому-либо может понравиться водить такую машину. 

Интересно что Эрих Фромм после исследования проблеммы отчуждения человека от общества и способов преодоления этого отчуждения, пришел к выводу что важнейшим из факторов являеться любовь в широком смысле слова. С моей точки зрения между матерью и сыном нет любви, они умеют рассуждать только о материальных, а не о духовных ценностях. Именно этим объясняются те трудности которые они испытывают в совместной жизни.

\subsubsection{\underline{Восприятие запахов}}

Мне также было интересно рассмотреть какой-нибудь из феноменов восприятия связанный с контактными рецепторами. Обоняние позволяет человеку ориентироваться относительно удаленных объектов, поэтому теоретически его можно отнести и к дистанционным рецепторам, однако на практике его чаще всего считают контактным способом восприятия, поскольку обонятельные ощущения возникают только при непосредственном воздействии молекул пахучего вещества на клетки рецепторов в носу человека. 

Когда Гарольд в первый раз приезжает в гости к Мод, она подводит его к "ароматизатору" и предлагает ему познакомиться с запахом снегопада. Когда он надевает маску, и начинает вдыхать запах, Мод спрашивает у него - "Что ты чувствуешь?". В ответ Гарольд делает четыре глубоких вдоха и после каждого из них у него складывается новое впечатление. В порядке в фильме он говорит: подземка, духи, сигареты, снегопад. Он постепенно воспринимает четыре различных стимула. Запахи накладываются друг на друга, и благодаря аддитивному восприятию у него в голове возникает запах и образ снегопада. Конечно нет такого четкого понятия как запах снегопада, однако благодаря воображению и умелому подбору других стимулов, можно создать образ того чего на самом деле нет. 

Интересно что Х. Хеннинг выделял лишь 6 основных типов запахов основываясь на расположении осмофорных групп (химического соединения):  пряный, фруктовый, цветочный, смолистый, гнилостный, пригорелый.

\begin{figure}[!h]
	\centering
	\includegraphics[scale=0.7]{Henning_6_smell_types}
	\caption{Классификация запахов по Хеннингу}
\end{figure}

\newpage

\subsubsection{\underline{5 (или 7) стадий переживания горя}}

Стоить заметить, что в реальной жизни эти фазы могут длиться довольно продолжительное время. Тем не менее режиссеру и актерам удается показать, как личность Гарольда проходит через эти стадии за гораздо более короткий период.  По своему опыту я бы сказал, что человек обязательно проходит через эти фазы, даже в мимолетных расстройствах.

Мод уже давно решила уйти из жизни сознательно. На своем дне рождения она открывает Гарольду, что она уже приняла критическую дозу снотворного и после полуночи ее уже не будет в живых. Гарольд еще молод и в этот вечер он собирался сделать ей предложение, поэтому ему трудно поверить в происходящее. Только после нескольких секунд шока и полнейшего непонимания, он вызывает скорую помощь. Таким образом он проходит через фазу отрицания. 

После этого обычно наступает фаза гнева, когда человек понимает ценность того, что он утратил и сила этой утраты заставляет его злиться на весь мир, который не переживает ее вместе с ним. У него появляется соблазн найти виноватого в этой потере. В фильме мы видим как несмотря на то что Гарольд любит Мод и уже научился мыслить подобно ей, ему трудно принять ее спокойное отношение к смерти. В сердцах он обращается к Мод за объяснением зачем она так поступила. 

Следующей стадией переживания горя считается компромисс. На этом этапе человеку приходит в голову мысль, что все еще можно исправить, что, если он очень захочет или достаточно сильно попросит, то все вернется на круги своя. Действительно, в госпитале Гарольд отчаянно пытается хоть как-то ускорить процесс у стойки регистрации, достает бумажник, показывая что готов отдать любую сумму за лечение. 

Предпоследняя стадия это депрессия. На этом этапе происходит очередной виток переживания горя, когда уже не остается надежды, проходит гнев и бесполезно искупление. Это считается самым долгим и тяжелым этапом переживания скорби. Человека перестает интересовать, что происходит вокруг, на него накатывает апатия и безразличие к жизни. В фильме Гарольд проводит ночь в госпитале, внутренне молясь и надеясь на то что доктора спасут Мод. Однако она умирает. Опустошенный и в отчаянии он садиться в машину и начинает гнать куда глаза глядят, раздумывая о возможности реального самоубийства. И вот он разгоняет автомобиль и направляет его в сторону обрыва. Хотя машина срывается вниз, мы видим как Гарольд вываливается из нее в последний момент. Это говорит о том, что он сумел совладать со своей скорбью. 

Итак, в последней стадии принятия человек смиряется со своей утратой и начинает действовать все более осознанно и целенаправленно, возвращаясь к жизни. Взгляд человека все больше обращается не в прошлое, а в будущее. Строятся новые планы, возвращается вкус к жизни, она вновь наполняется яркими красками. Мы видим как Гарольд берет свое банджо и начинает играть, а потом и пританцовывать. 

В некоторых случаях имеет смысл разбивать стадию отрицания на две фазы: шок, и собственно, отрицание. Также из стадии компромисса можно выделить стадию стыда, когда индивид берет на себя (возможно неоправданно) ответственность за приключившуюся беду.



\subsubsection{Заключение}
В работе выявленны следующие феномены: автономии и взросления, авторитарности, рыночного менталитета, восприятия запахов и 5 (или 7) стадий переживания горя. Более общие феномены, такие как автономия или авторитарность пересекаются между собой на протяжении всего фильма. Наличие одного из них, создает условия для проявления другого. 

В процессе анализа фильма мне удалось узнать много нового.  Например, о важности детского возраста в развитии человека, как это описанно З. Фрейдом. А также и об оппозиции такому  взгляду, например в лице Л. С. Выготского, не могу не процитировать его: "Для Фрейда человек, как каторжник к тачке, прикован к своему прошлому. Вся жизнь определяется в раннем детстве из элементарных комбинаций и вся без остатка сводится к изживанию детских конфликтов". К тому же я узнал о важности социума в развитии личности, и ознакомится с трудностями психического лечения человека, который не хочет или не готов к принятию чужой помощи.


\end{document}
