%=========================================
\subsection{Минимальные параметры технических средств}
Для испытания программы необходимо учесть следующие системные требования:
\begin{my_enumerate}
\item Компьютер, оснащенный:
    \begin{my_enumerate}
    \item Обязательно 64-разрядный (x64) процессор с тактовой частотой 1 гигагерц (ГГц) или выше;
    \item 1 ГБ оперативной памяти (ОЗУ);
    \item 1.5 ГБ свободного места на жестком диске;
    \item графическое устройство OpenGL с драйвером версии 3.1 или выше.
    \end{my_enumerate}
\item Монитор
\item Видеокарта
\item Мышь
\item Клавиатура
\end{my_enumerate}


%=========================================
\subsection{Минимальные программные средства}
Приложению необходим компьютер с поддержкой OpenGL версии 3.1. 64-битная операционная система Windows 7 или более поздняя версия Windows. Должен быть установлен .NET Framework версии 4.5.1, а также библиотеки Assimp версии 3.1 и OpenTK версии 1.1.4.

\subsection{Численность и калификация персонала}
Минимальное количество персонала, требуемого для работы программы: 1 оператор. Пользователь программы должен иметь образование не ниже среднего, обладать практическими навыками работы с компьютером и базовыми знаниями следующих понятий из сферы трех мерного моделирования: кость, корневая вершина (англ. root node), материал (англ. material), нормаль (англ. normal).
