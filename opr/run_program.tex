\subsection{Загрузка файлов}
Для загрузки данных из формата коллада (collada или .dae) необходимо выбрать его в меню:

\begin{figure}[h!]
    \centering
    \framebox[1cm][l]{never mind, so am I}
    %\makebox[2pt][c]{My image}
    %\includegraphics[width=0.8\textwidth]{image.png}
    \caption{Awesome Image}
    \label{fig:awesome_image}
\end{figure}

После загрузки файла, его имя будет добавиленно в список недавно открытых файлов "Recent Files".

\begin{figure}[h!]
    \centering
    \framebox[1cm][l]{never mind, so am I}
    %\makebox[2pt][c]{My image}
    %\includegraphics[width=0.8\textwidth]{image.png}
    \caption{Awesome Image}
    \label{fig:awesome_image}
\end{figure}


\subsection{Изменение положения камеры}
Изменять ракурс и приближение камеры можно при помощи мышки. Для приближения/отдаления используется колесо мышки. Движение с зажатым колесом двигает объект (только для камеры орбитального типа).


\subsection{Просмотр иерархии костей}
Структура загруженных данных отображена в виде дерева на панели справа. Выделенная на данный момент кость подсвеченна ярко-синим цветом.

\begin{figure}[h!]
    \centering
    \framebox[1cm][l]{never mind, so am I}
    %\makebox[2pt][c]{My image}
    %\includegraphics[width=0.8\textwidth]{image.png}
    \caption{Awesome Image}
    \label{fig:awesome_image}
\end{figure}



\subsection{Изменение параметров отрисовки и анимации}
Элемент ScrollBar показывает текущий момент в анимации и предоставляет 
возможность перейти к любому моменту времени. Также есть панель для настоек работы программы позволяющая изменять следующие параметры:
\begin{my_enumerate}
\item Выбор между двумя видами камер в OpenGL (скованой орбитой и свободной).
\item Вкл./Выкл. воспроизведения анимации.
\item Вкл./Выкл. отрисовку учитывая нормали данной модели.
\item Вкл./Выкл. отрисовки материала данной модели.
\item Отрисовка всех костей.
\end{my_enumerate}


\begin{figure}[h!]
    \centering
    \framebox[1cm][l]{never mind, so am I}
    %\makebox[2pt][c]{My image}
    %\includegraphics[width=0.8\textwidth]{image.png}
    \caption{Awesome Image}
    \label{fig:awesome_image}
\end{figure}


\subsection{Всплывающие окна}
В случае если выбран файл не соответствующий требованиям входных данных отображается всплывающее окно:
\begin{figure}[h!]
    \centering
    \framebox[1cm][l]{never mind, so am I}
    %\makebox[2pt][c]{My image}
    %\includegraphics[width=0.8\textwidth]{image.png}
    \caption{Awesome Image}
    \label{fig:awesome_image}
\end{figure}


\subsection{Завершение работы с программой}

При нажатии на кнопку "Закрыть" в правом верхнем углу программы.
