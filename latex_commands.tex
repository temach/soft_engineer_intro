\documentclass[a4paper,10pt]{article} % формат бумаги А4, шрифт по умолчанию - 12pt

% заметь, что в квадратных скобках вводятся необязательные аргументы пакетов.
% а в фигурных - обязательные

\usepackage[T2A]{fontenc} % поддержка кириллицы в Латехе
\usepackage[utf8]{inputenc} % включаю кодировку ютф8
\usepackage[english,russian]{babel} % использую русский и английский языки с переносами

\usepackage{indentfirst} % делать отступ в начале параграфа
\usepackage{amsmath} % математические штуковины
\usepackage{mathtools} % еще математические штуковины
\usepackage{mathtext}
\usepackage{multicol} % подключаю мультиколоночность в тексте
\usepackage{graphicx} % пакет для вставки графики, я хз нахуя он нужен в этом документе
\usepackage{listings} % пакет для вставки кода
\usepackage[table]{xcolor}% http://ctan.org/pkg/xcolor for coloring the inside of a cell
\usepackage{colortbl}
\usepackage[lofdepth,lotdepth]{subfig} % so we can place figures side by side

\usepackage{geometry} % меняю поля страницы
\usepackage{caption}

%из параметров ниже понятно, какие части полей страницы меняются:
\geometry{left=1.4cm}
\geometry{right=2cm}
\geometry{top=1.5cm}
\geometry{bottom=2cm}

\renewcommand{\baselinestretch}{1} % меняю ширину между строками на 1.5
\righthyphenmin=2


\definecolor{lightgray}{gray}{0.9}
\definecolor{goodrow}{rgb}{0.7,1,0.7}

\begin{document}


\begin{titlepage}
\newpage

\begin{center}
{\large НАЦИОНАЛЬНЫЙ ИССЛЕДОВАТЕЛЬСКИЙ УНИВЕРСИТЕТ \\
«ВЫСШАЯ ШКОЛА ЭКОНОМИКИ» \\
Дисциплина: «Дискретная математика»}

\vfill % заполняет длину страницы вертикально

{\large Домашнее задание 3}

\bigskip

\underline{ Исследование алгоритмов решения} \\
\underline{ метрической задачи коммивояжера} \\

\bigskip

Вариант 002 \\

\vfill

\begin{flushright}
Выполнил: Абрамов Артем,\\
студент группы БПИ1511\medskip \\
Преподаватель: Авдошин С.М., \\
профессор департамента \\
программной инженерии \\
факультета компьютерных наук
\end{flushright}

\vfill

Москва \number\year

\end{center}
\end{titlepage}

% my matrix!
%***,   3,   9,   8,  11,   5
%  3, ***,   8,   7,   8,   4
%  9,   8, ***,   3,   8,   4
%  8,   7,   3, ***,  11,   3
% 11,   8,   8,  11, ***,  10
%  5,   4,   4,   3,  10, ***


1. \quad На плоскости задано множество точек $V = {1, 2, 3, 4, 5, 6}$ своими координатами
$(x = 4,y = 1),(x = 4,y = 3),(x = 2,y = 7),(x = 9,y = 6),(x = 10,y = 7),(x = 6,y = 10)$ \\

\smallskip

2. \quad Вычислим элементы $d_{ij}$ весовой матрицы смежности графа 
$G = <V, V*V>$ по формуле $dij = |xi - xj|  +  |yi - yj|$

\begin{flushleft}\begin{tabular}[]{|c|c|c|c|c|c|c|}
\hline
  0 &    3 &    9 &    8 &   11 &    5 \\
\hline
  3 &    0 &    8 &    7 &    8 &    4 \\
\hline
  9 &    8 &    0 &    3 &    8 &    4 \\
\hline
  8 &    7 &    3 &    0 &   11 &    3 \\
\hline
 11 &    8 &    8 &   11 &    0 &   10 \\
\hline
  5 &    4 &    4 &    3 &   10 &    0 \\
\hline
\end{tabular}
\end{flushleft}


\smallskip

3. \quad Используя метод ветвей и границ, найдем множество кодов всех оптимальных гамильтоновых
 циклов являющихся решением задачи коммивояжера на графе G.
Петли не могут быть частью решения. Поэтому положим диагональные элементы равными бесконечности.

\begin{flushleft}\begin{tabular}[]{|c|c|c|c|c|c|c|}
\hline
$\infty$ &    3 &    9 &    8 &   11 &    5 \\
\hline
  3 &  $\infty$ &    8 &    7 &    8 &    4 \\
\hline
  9 &    8 &  $\infty$ &    3 &    8 &    4 \\
\hline
  8 &    7 &    3 &  $\infty$ &   11 &    3 \\
\hline
 11 &    8 &    8 &   11 &  $\infty$ &   10 \\
\hline
  5 &    4 &    4 &    3 &   10 &  $\infty$ \\
\hline
\end{tabular}
\end{flushleft}


Обозначим через
$ S = \{ p:V \rightarrow V \} (p(l) = l) \& ( \forall j \subset V ) ( \forall j \subset V) ((p(i) = p(j)) \Longrightarrow (i = j))$
множество кодов всех гамильтоновых циклов $v=(p_1,p_2,p_3,p_4,p_5,p_6,p_1)$ графа G, 
заданного весовой матрицей смежности D. 
Здесь $p_{i}$ используется в качестве сокращенной записи $p(i)$.

Найдем нижнюю границу $b$ множества $S$


% make the captions stick to the LEFT of the page
\captionsetup{justification=raggedright,
singlelinecheck=false
}

\captionsetup[subfloat]{labelformat=empty}


\begin{table}[ht]
\subfloat[][]{\begin{tabular}[]{|>{\columncolor[gray]{0.9}}c|c|c|c|c|c|c|c|c|}
\hline
\rowcolor{lightgray}  s & 1 & 2 & 3 & 4 & 5 & 6 & min $\alpha$ \\
\hline
1     & $\infty$ &    3 &    9 &    8 &   11 &    5 & 3 \\
\hline
2     &   3 &  $\infty$ &    8 &    7 &    8 &    4 & 3 \\
\hline
3     &   9 &    8 &  $\infty$ &    3 &    8 &    4 & 3 \\
\hline
4     &   8 &    7 &    3 &  $\infty$ &   11 &    3 & 3 \\
\hline
5     &  11 &    8 &    8 &   11 &  $\infty$ &   10 & 8 \\
\hline
6     &   5 &    4 &    4 &    3 &   10 &  $\infty$ & 3 \\
\hline
\end{tabular}
}\hfill
\subfloat[][]{\begin{tabular}[]{|>{\columncolor[gray]{0.9}}c|c|c|c|c|c|c|c|}
\hline
s & 1 & 2 & 3 & 4 & 5 & 6 \\
\hline
1     & $\infty$ &    0 &    6 &    5 &   8 &    2  \\
\hline
2     &   0 &  $\infty$ &    5 &    4 &    5 &    1 \\
\hline
3     &   6 &    5 &  $\infty$ &    0 &    5 &    1 \\
\hline
4     &   5 &    4 &    0 &  $\infty$ &   8 &    0  \\
\hline
5     &  3 &    0 &    0 &   3 &  $\infty$ &   2    \\
\hline
6     &   2 &    1 &    1 &    0 &   7 &  $\infty$  \\
\hline
min $\beta$ &  0 &      0 &      0 &      0 &      5 &     0 \\
\hline
\end{tabular}
}\captionof*{table}{$b = \alpha + \beta = 28$}
\end{table}

\newpage

Определим дугу ветвления для разбиения множества  s\\
\begin{flushleft}\begin{tabular}[]{|>{\columncolor[gray]{0.9}}c|c|c|c|c|c|c|}
\hline
\rowcolor{lightgray} s & 1 & 2 & 3 & 4 & 5 & 6\\
\hline
1 & $\infty$ &      0 &      6 &      5 &      3 &      2\\
\hline
2 &      0 & $\infty$ &      5 &      4 &      0 &      1\\
\hline
3 &      6 &      5 & $\infty$ &      0 &      0 &      1\\
\hline
4 &      5 &      4 &      0 & $\infty$ &      3 &      0\\
\hline
5 &      3 &      0 &      0 &      3 & $\infty$ &      2\\
\hline
6 &      2 &      1 &      1 &      0 &      2 & $\infty$\\
\hline
\end{tabular}
\captionof*{table}{(1,2)}
\end{flushleft}


\begin{table}[ht]
\subfloat[][]{\begin{tabular}[]{|>{\columncolor[gray]{0.9}}c|c|c|c|c|c|c|c}
\hline
\rowcolor{lightgray}  s0 & 1 & 2 & 3 & 4 & 5 & 6 & min\\
\hline
1 & $\infty$ & \cellcolor{yellow}$\infty$ &      6 &      5 &      3 &      2 & 2\\
\hline
2 &      0 & $\infty$ &      5 &      4 &      0 &      1 & 0\\
\hline
3 &      6 &      5 & $\infty$ &      0 &      0 &      1 & 0\\
\hline
4 &      5 &      4 &      0 & $\infty$ &      3 &      0 & 0\\
\hline
5 &      3 &      0 &      0 &      3 & $\infty$ &      2 & 0\\
\hline
6 &      2 &      1 &      1 &      0 &      2 & $\infty$ & 0\\
\hline
min &      0 &      0 &      0 &      0 &      0 &      0\\
\end{tabular}
}\hfill
\subfloat[][]{\begin{tabular}[]{|>{\columncolor[gray]{0.9}}c|c|c|c|c|c|c|}
\hline
\rowcolor{lightgray}  s0 & 1 & 2 & 3 & 4 & 5 & 6\\
\hline
1 & $\infty$ & $\infty$ &      4 &      3 &      1 &      0\\
\hline
2 &      0 & $\infty$ &      5 &      4 &      0 &      1\\
\hline
3 &      6 &      5 & $\infty$ &      0 &      0 &      1\\
\hline
4 &      5 &      4 &      0 & $\infty$ &      3 &      0\\
\hline
5 &      3 &      0 &      0 &      3 & $\infty$ &      2\\
\hline
6 &      2 &      1 &      1 &      0 &      2 & $\infty$\\
\hline
\end{tabular}
}\captionof*{table}{b0 = b + 2 + 0 = 30}
\end{table}


\begin{table}[ht]
\subfloat[][]{\begin{tabular}[]{|>{\columncolor[gray]{0.9}}c|c|c|c|c|c|c}
\hline
\rowcolor{lightgray}  s1 & 1 & 3 & 4 & 5 & 6 & min\\
\hline
2 & \cellcolor{yellow}$\infty$ &      5 &      4 &      0 &      1 & 0\\
\hline
3 &      6 & $\infty$ &      0 &      0 &      1 & 0\\
\hline
4 &      5 &      0 & $\infty$ &      3 &      0 & 0\\
\hline
5 &      3 &      0 &      3 & $\infty$ &      2 & 0\\
\hline
6 &      2 &      1 &      0 &      2 & $\infty$ & 0\\
\hline
min &      2 &      0 &      0 &      0 &      0\\
\end{tabular}
}\hfill
\subfloat[][]{\begin{tabular}[]{|>{\columncolor[gray]{0.9}}c|c|c|c|c|c|}
\hline
\rowcolor{lightgray}  s1 & 1 & 3 & 4 & 5 & 6\\
\hline
2 & $\infty$ &      5 &      4 &      0 &      1\\
\hline
3 &      4 & $\infty$ &      0 &      0 &      1\\
\hline
4 &      3 &      0 & $\infty$ &      3 &      0\\
\hline
5 &      1 &      0 &      3 & $\infty$ &      2\\
\hline
6 &      0 &      1 &      0 &      2 & $\infty$\\
\hline
\end{tabular}
}\captionof*{table}{b1 = b + 0 + 2 = 30}
\end{table}

\newpage

Определим дугу ветвления для разбиения множества  s1\\
\begin{flushleft}\begin{tabular}[]{|>{\columncolor[gray]{0.9}}c|c|c|c|c|c|}
\hline
\rowcolor{lightgray}  s1 & 1 & 3 & 4 & 5 & 6\\
\hline
2 & $\infty$ &      5 &      4 &      0 &      1\\
\hline
3 &      4 & $\infty$ &      0 &      0 &      1\\
\hline
4 &      3 &      0 & $\infty$ &      3 &      0\\
\hline
5 &      1 &      0 &      3 & $\infty$ &      2\\
\hline
6 &      0 &      1 &      0 &      2 & $\infty$\\
\hline
\end{tabular}
\captionof*{table}{(2,5)}
\end{flushleft}


\begin{table}[ht]
\subfloat[][]{\begin{tabular}[]{|>{\columncolor[gray]{0.9}}c|c|c|c|c|c|c}
\hline
\rowcolor{lightgray}  s10 & 1 & 3 & 4 & 5 & 6 & min\\
\hline
2 & $\infty$ &      5 &      4 & \cellcolor{yellow}$\infty$ &      1 & 1\\
\hline
3 &      4 & $\infty$ &      0 &      0 &      1 & 0\\
\hline
4 &      3 &      0 & $\infty$ &      3 &      0 & 0\\
\hline
5 &      1 &      0 &      3 & $\infty$ &      2 & 0\\
\hline
6 &      0 &      1 &      0 &      2 & $\infty$ & 0\\
\hline
min &      0 &      0 &      0 &      0 &      0\\
\end{tabular}
}\hfill
\subfloat[][]{\begin{tabular}[]{|>{\columncolor[gray]{0.9}}c|c|c|c|c|c|}
\hline
\rowcolor{lightgray}  s10 & 1 & 3 & 4 & 5 & 6\\
\hline
2 & $\infty$ &      4 &      3 & $\infty$ &      0\\
\hline
3 &      4 & $\infty$ &      0 &      0 &      1\\
\hline
4 &      3 &      0 & $\infty$ &      3 &      0\\
\hline
5 &      1 &      0 &      3 & $\infty$ &      2\\
\hline
6 &      0 &      1 &      0 &      2 & $\infty$\\
\hline
\end{tabular}
}\captionof*{table}{b10 = b1 + 1 + 0 = 31}
\end{table}


\begin{table}[ht]
\subfloat[][]{\begin{tabular}[]{|>{\columncolor[gray]{0.9}}c|c|c|c|c|c}
\hline
\rowcolor{lightgray}  s11 & 1 & 3 & 4 & 6 & min\\
\hline
3 &      4 & $\infty$ &      0 &      1 & 0\\
\hline
4 &      3 &      0 & $\infty$ &      0 & 0\\
\hline
5 & \cellcolor{yellow}$\infty$ &      0 &      3 &      2 & 0\\
\hline
6 &      0 &      1 &      0 & $\infty$ & 0\\
\hline
min &      0 &      0 &      0 &      0\\
\end{tabular}
}\hfill
\subfloat[][]{\begin{tabular}[]{|>{\columncolor[gray]{0.9}}c|c|c|c|c|}
\hline
\rowcolor{lightgray}  s11 & 1 & 3 & 4 & 6\\
\hline
3 &      4 & $\infty$ &      0 &      1\\
\hline
4 &      3 &      0 & $\infty$ &      0\\
\hline
5 & $\infty$ &      0 &      3 &      2\\
\hline
6 &      0 &      1 &      0 & $\infty$\\
\hline
\end{tabular}
}\captionof*{table}{b11 = b1 + 0 + 0 = 30}
\end{table}

\newpage

Определим дугу ветвления для разбиения множества  s0\\
\begin{flushleft}\begin{tabular}[]{|>{\columncolor[gray]{0.9}}c|c|c|c|c|c|c|}
\hline
\rowcolor{lightgray}  s0 & 1 & 2 & 3 & 4 & 5 & 6\\
\hline
1 & $\infty$ & $\infty$ &      4 &      3 &      1 &      0\\
\hline
2 &      0 & $\infty$ &      5 &      4 &      0 &      1\\
\hline
3 &      6 &      5 & $\infty$ &      0 &      0 &      1\\
\hline
4 &      5 &      4 &      0 & $\infty$ &      3 &      0\\
\hline
5 &      3 &      0 &      0 &      3 & $\infty$ &      2\\
\hline
6 &      2 &      1 &      1 &      0 &      2 & $\infty$\\
\hline
\end{tabular}
\captionof*{table}{(2,1)}
\end{flushleft}


\begin{table}[ht]
\subfloat[][]{\begin{tabular}[]{|>{\columncolor[gray]{0.9}}c|c|c|c|c|c|c|c}
\hline
\rowcolor{lightgray}  s00 & 1 & 2 & 3 & 4 & 5 & 6 & min\\
\hline
1 & $\infty$ & $\infty$ &      4 &      3 &      1 &      0 & 0\\
\hline
2 & \cellcolor{yellow}$\infty$ & $\infty$ &      5 &      4 &      0 &      1 & 0\\
\hline
3 &      6 &      5 & $\infty$ &      0 &      0 &      1 & 0\\
\hline
4 &      5 &      4 &      0 & $\infty$ &      3 &      0 & 0\\
\hline
5 &      3 &      0 &      0 &      3 & $\infty$ &      2 & 0\\
\hline
6 &      2 &      1 &      1 &      0 &      2 & $\infty$ & 0\\
\hline
min &      2 &      0 &      0 &      0 &      0 &      0\\
\end{tabular}
}\hfill
\subfloat[][]{\begin{tabular}[]{|>{\columncolor[gray]{0.9}}c|c|c|c|c|c|c|}
\hline
\rowcolor{lightgray}  s00 & 1 & 2 & 3 & 4 & 5 & 6\\
\hline
1 & $\infty$ & $\infty$ &      4 &      3 &      1 &      0\\
\hline
2 & $\infty$ & $\infty$ &      5 &      4 &      0 &      1\\
\hline
3 &      4 &      5 & $\infty$ &      0 &      0 &      1\\
\hline
4 &      3 &      4 &      0 & $\infty$ &      3 &      0\\
\hline
5 &      1 &      0 &      0 &      3 & $\infty$ &      2\\
\hline
6 &      0 &      1 &      1 &      0 &      2 & $\infty$\\
\hline
\end{tabular}
}\captionof*{table}{b00 = b0 + 0 + 2 = 32}
\end{table}


\begin{table}[ht]
\subfloat[][]{\begin{tabular}[]{|>{\columncolor[gray]{0.9}}c|c|c|c|c|c|c}
\hline
\rowcolor{lightgray}  s01 & 2 & 3 & 4 & 5 & 6 & min\\
\hline
1 & $\infty$ &      4 &      3 &      1 &      0 & 0\\
\hline
3 &      5 & $\infty$ &      0 &      0 &      1 & 0\\
\hline
4 &      4 &      0 & $\infty$ &      3 &      0 & 0\\
\hline
5 &      0 &      0 &      3 & $\infty$ &      2 & 0\\
\hline
6 &      1 &      1 &      0 &      2 & $\infty$ & 0\\
\hline
min &      0 &      0 &      0 &      0 &      0\\
\end{tabular}
}\hfill
\subfloat[][]{\begin{tabular}[]{|>{\columncolor[gray]{0.9}}c|c|c|c|c|c|}
\hline
\rowcolor{lightgray}  s01 & 2 & 3 & 4 & 5 & 6\\
\hline
1 & $\infty$ &      4 &      3 &      1 &      0\\
\hline
3 &      5 & $\infty$ &      0 &      0 &      1\\
\hline
4 &      4 &      0 & $\infty$ &      3 &      0\\
\hline
5 &      0 &      0 &      3 & $\infty$ &      2\\
\hline
6 &      1 &      1 &      0 &      2 & $\infty$\\
\hline
\end{tabular}
}\captionof*{table}{b01 = b0 + 0 + 0 = 30}
\end{table}

\newpage

Определим дугу ветвления для разбиения множества  s11\\
\begin{flushleft}\begin{tabular}[]{|>{\columncolor[gray]{0.9}}c|c|c|c|c|}
\hline
\rowcolor{lightgray}  s11 & 1 & 3 & 4 & 6\\
\hline
3 &      4 & $\infty$ &      0 &      1\\
\hline
4 &      3 &      0 & $\infty$ &      0\\
\hline
5 & $\infty$ &      0 &      3 &      2\\
\hline
6 &      0 &      1 &      0 & $\infty$\\
\hline
\end{tabular}
\captionof*{table}{(6,1)}
\end{flushleft}


\begin{table}[ht]
\subfloat[][]{\begin{tabular}[]{|>{\columncolor[gray]{0.9}}c|c|c|c|c|c}
\hline
\rowcolor{lightgray}  s110 & 1 & 3 & 4 & 6 & min\\
\hline
3 &      4 & $\infty$ &      0 &      1 & 0\\
\hline
4 &      3 &      0 & $\infty$ &      0 & 0\\
\hline
5 & $\infty$ &      0 &      3 &      2 & 0\\
\hline
6 & \cellcolor{yellow}$\infty$ &      1 &      0 & $\infty$ & 0\\
\hline
min &      3 &      0 &      0 &      0\\
\end{tabular}
}\hfill
\subfloat[][]{\begin{tabular}[]{|>{\columncolor[gray]{0.9}}c|c|c|c|c|}
\hline
\rowcolor{lightgray}  s110 & 1 & 3 & 4 & 6\\
\hline
3 &      1 & $\infty$ &      0 &      1\\
\hline
4 &      0 &      0 & $\infty$ &      0\\
\hline
5 & $\infty$ &      0 &      3 &      2\\
\hline
6 & $\infty$ &      1 &      0 & $\infty$\\
\hline
\end{tabular}
}\captionof*{table}{b110 = b11 + 0 + 3 = 33}
\end{table}


\begin{table}[ht]
\subfloat[][]{\begin{tabular}[]{|>{\columncolor[gray]{0.9}}c|c|c|c|c}
\hline
\rowcolor{lightgray}  s111 & 3 & 4 & 6 & min\\
\hline
3 & $\infty$ &      0 &      1 & 0\\
\hline
4 &      0 & $\infty$ &      0 & 0\\
\hline
5 &      0 &      3 & \cellcolor{yellow}$\infty$ & 0\\
\hline
min &      0 &      0 &      0\\
\end{tabular}
}\hfill
\subfloat[][]{\begin{tabular}[]{|>{\columncolor[gray]{0.9}}c|c|c|c|}
\hline
\rowcolor{lightgray}  s111 & 3 & 4 & 6\\
\hline
3 & $\infty$ &      0 &      1\\
\hline
4 &      0 & $\infty$ &      0\\
\hline
5 &      0 &      3 & $\infty$\\
\hline
\end{tabular}
}\captionof*{table}{b111 = b11 + 0 + 0 = 30}
\end{table}

\newpage


Определим дугу ветвления для разбиения множества  s01\\
\begin{flushleft}\begin{tabular}[]{|>{\columncolor[gray]{0.9}}c|c|c|c|c|c|}
\hline
\rowcolor{lightgray}  s01 & 2 & 3 & 4 & 5 & 6\\
\hline
1 & $\infty$ &      4 &      3 &      1 &      0\\
\hline
3 &      5 & $\infty$ &      0 &      0 &      1\\
\hline
4 &      4 &      0 & $\infty$ &      3 &      0\\
\hline
5 &      0 &      0 &      3 & $\infty$ &      2\\
\hline
6 &      1 &      1 &      0 &      2 & $\infty$\\
\hline
\end{tabular}
\captionof*{table}{(1,6)}
\end{flushleft}


\begin{table}[ht]
\subfloat[][]{\begin{tabular}[]{|>{\columncolor[gray]{0.9}}c|c|c|c|c|c|c}
\hline
\rowcolor{lightgray}  s010 & 2 & 3 & 4 & 5 & 6 & min\\
\hline
1 & $\infty$ &      4 &      3 &      1 & \cellcolor{yellow}$\infty$ & 1\\
\hline
3 &      5 & $\infty$ &      0 &      0 &      1 & 0\\
\hline
4 &      4 &      0 & $\infty$ &      3 &      0 & 0\\
\hline
5 &      0 &      0 &      3 & $\infty$ &      2 & 0\\
\hline
6 &      1 &      1 &      0 &      2 & $\infty$ & 0\\
\hline
min &      0 &      0 &      0 &      0 &      0\\
\end{tabular}
}\hfill
\subfloat[][]{\begin{tabular}[]{|>{\columncolor[gray]{0.9}}c|c|c|c|c|c|}
\hline
\rowcolor{lightgray}  s010 & 2 & 3 & 4 & 5 & 6\\
\hline
1 & $\infty$ &      3 &      2 &      0 & $\infty$\\
\hline
3 &      5 & $\infty$ &      0 &      0 &      1\\
\hline
4 &      4 &      0 & $\infty$ &      3 &      0\\
\hline
5 &      0 &      0 &      3 & $\infty$ &      2\\
\hline
6 &      1 &      1 &      0 &      2 & $\infty$\\
\hline
\end{tabular}
}\captionof*{table}{b010 = b01 + 1 + 0 = 31}
\end{table}


\begin{table}[ht]
\subfloat[][]{\begin{tabular}[]{|>{\columncolor[gray]{0.9}}c|c|c|c|c|c}
\hline
\rowcolor{lightgray}  s011 & 2 & 3 & 4 & 5 & min\\
\hline
3 &      5 & $\infty$ &      0 &      0 & 0\\
\hline
4 &      4 &      0 & $\infty$ &      3 & 0\\
\hline
5 &      0 &      0 &      3 & $\infty$ & 0\\
\hline
6 & \cellcolor{yellow}$\infty$ &      1 &      0 &      2 & 0\\
\hline
min &      0 &      0 &      0 &      0\\
\end{tabular}
}\hfill
\subfloat[][]{\begin{tabular}[]{|>{\columncolor[gray]{0.9}}c|c|c|c|c|}
\hline
\rowcolor{lightgray}  s011 & 2 & 3 & 4 & 5\\
\hline
3 &      5 & $\infty$ &      0 &      0\\
\hline
4 &      4 &      0 & $\infty$ &      3\\
\hline
5 &      0 &      0 &      3 & $\infty$\\
\hline
6 & $\infty$ &      1 &      0 &      2\\
\hline
\end{tabular}
}\captionof*{table}{b011 = b01 + 0 + 0 = 30}
\end{table}

\newpage


Определим дугу ветвления для разбиения множества  s111\\
\begin{flushleft}\begin{tabular}[]{|>{\columncolor[gray]{0.9}}c|c|c|c|}
\hline
\rowcolor{lightgray}  s111 & 3 & 4 & 6\\
\hline
3 & $\infty$ &      0 &      1\\
\hline
4 &      0 & $\infty$ &      0\\
\hline
5 &      0 &      3 & $\infty$\\
\hline
\end{tabular}
\captionof*{table}{(3,4)}
\end{flushleft}


\begin{table}[ht]
\subfloat[][]{\begin{tabular}[]{|>{\columncolor[gray]{0.9}}c|c|c|c|c}
\hline
\rowcolor{lightgray}  s1110 & 3 & 4 & 6 & min\\
\hline
3 & $\infty$ & \cellcolor{yellow}$\infty$ &      1 & 1\\
\hline
4 &      0 & $\infty$ &      0 & 0\\
\hline
5 &      0 &      3 & $\infty$ & 0\\
\hline
min &      0 &      3 &      0\\
\end{tabular}
}\hfill
\subfloat[][]{\begin{tabular}[]{|>{\columncolor[gray]{0.9}}c|c|c|c|}
\hline
\rowcolor{lightgray}  s1110 & 3 & 4 & 6\\
\hline
3 & $\infty$ & $\infty$ &      0\\
\hline
4 &      0 & $\infty$ &      0\\
\hline
5 &      0 &      0 & $\infty$\\
\hline
\end{tabular}
}\captionof*{table}{b1110 = b111 + 1 + 3 = 34}
\end{table}


\begin{tabular}[]{|>{\columncolor[gray]{0.9}}c|c|c|}
\hline
\rowcolor{lightgray}  s1111 & 3 & 6\\
\hline
4 & \cellcolor{yellow}$\infty$ &      0\\
\hline
5 &      0 & $\infty$\\
\hline
\end{tabular}
\captionof*{table}{ b1111 = b111 + 0 + 0 = 30\\ 
$v = \{ 1,2,5,3,4,6,1 \}$
}


\newpage


Определим дугу ветвления для разбиения множества  s011\\
\begin{flushleft}\begin{tabular}[]{|>{\columncolor[gray]{0.9}}c|c|c|c|c|}
\hline
\rowcolor{lightgray}  s011 & 2 & 3 & 4 & 5\\
\hline
3 &      5 & $\infty$ &      0 &      0\\
\hline
4 &      4 &      0 & $\infty$ &      3\\
\hline
5 &      0 &      0 &      3 & $\infty$\\
\hline
6 & $\infty$ &      1 &      0 &      2\\
\hline
\end{tabular}
\captionof*{table}{(5,2)}
\end{flushleft}


\begin{table}[ht]
\subfloat[][]{\begin{tabular}[]{|>{\columncolor[gray]{0.9}}c|c|c|c|c|c}
\hline
\rowcolor{lightgray}  s0110 & 2 & 3 & 4 & 5 & min\\
\hline
3 &      5 & $\infty$ &      0 &      0 & 0\\
\hline
4 &      4 &      0 & $\infty$ &      3 & 0\\
\hline
5 & \cellcolor{yellow}$\infty$ &      0 &      3 & $\infty$ & 0\\
\hline
6 & $\infty$ &      1 &      0 &      2 & 0\\
\hline
min &      4 &      0 &      0 &      0\\
\end{tabular}
}\hfill
\subfloat[][]{\begin{tabular}[]{|>{\columncolor[gray]{0.9}}c|c|c|c|c|}
\hline
\rowcolor{lightgray}  s0110 & 2 & 3 & 4 & 5\\
\hline
3 &      1 & $\infty$ &      0 &      0\\
\hline
4 &      0 &      0 & $\infty$ &      3\\
\hline
5 & $\infty$ &      0 &      3 & $\infty$\\
\hline
6 & $\infty$ &      1 &      0 &      2\\
\hline
\end{tabular}
}\captionof*{table}{b0110 = b011 + 0 + 4 = 34}
\end{table}


\begin{table}[ht]
\subfloat[][]{\begin{tabular}[]{|>{\columncolor[gray]{0.9}}c|c|c|c|c}
\hline
\rowcolor{lightgray}  s0111 & 3 & 4 & 5 & min\\
\hline
3 & $\infty$ &      0 &      0 & 0\\
\hline
4 &      0 & $\infty$ &      3 & 0\\
\hline
6 &      1 &      0 & \cellcolor{yellow}$\infty$ & 0\\
\hline
min &      0 &      0 &      0\\
\end{tabular}
}\hfill
\subfloat[][]{\begin{tabular}[]{|>{\columncolor[gray]{0.9}}c|c|c|c|}
\hline
\rowcolor{lightgray}  s0111 & 3 & 4 & 5\\
\hline
3 & $\infty$ &      0 &      0\\
\hline
4 &      0 & $\infty$ &      3\\
\hline
6 &      1 &      0 & $\infty$\\
\hline
\end{tabular}
}\captionof*{table}{b0111 = b011 + 0 + 0 = 30}
\end{table}

\newpage


Определим дугу ветвления для разбиения множества  s0111\\
\begin{flushleft}\begin{tabular}[]{|>{\columncolor[gray]{0.9}}c|c|c|c|}
\hline
\rowcolor{lightgray}  s0111 & 3 & 4 & 5\\
\hline
3 & $\infty$ &      0 &      0\\
\hline
4 &      0 & $\infty$ &      3\\
\hline
6 &      1 &      0 & $\infty$\\
\hline
\end{tabular}
\captionof*{table}{(3,5)}
\end{flushleft}


\begin{table}[ht]
\subfloat[][]{\begin{tabular}[]{|>{\columncolor[gray]{0.9}}c|c|c|c|c}
\hline
\rowcolor{lightgray}  s01110 & 3 & 4 & 5 & min\\
\hline
3 & $\infty$ &      0 & \cellcolor{yellow}$\infty$ & 0\\
\hline
4 &      0 & $\infty$ &      3 & 0\\
\hline
6 &      1 &      0 & $\infty$ & 0\\
\hline
min &      0 &      0 &      3\\
\end{tabular}
}\hfill
\subfloat[][]{\begin{tabular}[]{|>{\columncolor[gray]{0.9}}c|c|c|c|}
\hline
\rowcolor{lightgray}  s01110 & 3 & 4 & 5\\
\hline
3 & $\infty$ &      0 & $\infty$\\
\hline
4 &      0 & $\infty$ &      0\\
\hline
6 &      1 &      0 & $\infty$\\
\hline
\end{tabular}
}\captionof*{table}{b01110 = b0111 + 0 + 3 = 33}
\end{table}


\begin{tabular}[]{|>{\columncolor[gray]{0.9}}c|c|c|}
\hline
\rowcolor{lightgray}  s01111 & 3 & 4\\
\hline
4 &      0 & $\infty$\\
\hline
6 &  \cellcolor{yellow}$\infty$ &      0\\
\hline
\end{tabular}
\captionof*{table}{b01111 = b0111 + 0 + 0 = 30\\ 
$v = \{ 1,6,4,3,5,2,1 \}$
}

\newpage


\begin{figure}[!h]
    \centering
    \includegraphics[scale=0.7]{var_002}
\end{figure}

\bigskip

{\large
Ответ: множество кодов всех оптимальных гамильтоновых циклов являющихся решением задачи коммивояжера на графе G есть $\{1643521, 1253461 \}$. Вес $f_o$ оптимального гамильтонова цикла равен 30.
}

\newpage


4. \quad Найдем приближенное решение задачи коммивояжера $v_1$ с помощью первого алгоритма Кристофидеса. Используя весовую матрицу смежности D графа G , построим кратчайшее связывающее дерево T с помощю алгоритма Прима.

\medskip

% dark means == do not take


% LINE ONE TABLE ONE
\begin{tabular}[]{|c|c|c|c|c|c|c|}
\hline
$\infty$ &    \cellcolor{yellow}3 &    9 &    8 &   11 &    5 \\
\hline
  3 &  $\infty$ &    8 &    7 &    8 &    4 \\
\hline
  9 &    8 &  $\infty$ &    3 &    8 &    4 \\
\hline
  8 &    7 &    3 &  $\infty$ &   11 &    3 \\
\hline
 11 &    8 &    8 &   11 &  $\infty$ &   10 \\
\hline
  5 &    4 &    4 &    3 &   10 &  $\infty$ \\
\hline
\end{tabular}
\hspace{4cm}
% LINE ONE second TABLE
\begin{tabular}[]{|c|c|c|c|c|}
\hline
\rowcolor{goodrow}     9 &    8 &   11 &    5 \\
\hline
 \rowcolor{goodrow}     8 &    7 &    8 &    \cellcolor{yellow}4 \\
\hline
  $\infty$ &    3 &    8 &    4 \\
\hline
    3 &  $\infty$ &   11 &    3 \\
\hline
    8 &   11 &  $\infty$ &   10 \\
\hline
    4 &    3 &   10 &  $\infty$ \\
\hline
\end{tabular}

\medskip

% LINE TWO TABLE ONE
\begin{tabular}[]{|c|c|c|c|c|c|c|}
\hline
 \rowcolor{goodrow}  9 &    8 &   11 \\
\hline
 \rowcolor{goodrow} 8 &    7 &    8  \\
\hline
  $\infty$ &    3 &    8 \\
\hline
    3 &  $\infty$ &   11 \\
\hline
    8 &   11 &  $\infty$ \\
\hline
 \rowcolor{goodrow}    4 &    \cellcolor{yellow}3 &   10 \\
\hline
\end{tabular}
\hspace{7cm}
% LINE TWO second TABLE
\begin{tabular}[]{|c|c|c|c|c|c|c|}
\hline
 \rowcolor{goodrow}  9 &   11 \\
\hline
 \rowcolor{goodrow} 8 &    8  \\
\hline
  $\infty$ &    8 \\
\hline
\rowcolor{goodrow}   \cellcolor{yellow}3 &   11 \\
\hline
    8 &  $\infty$ \\
\hline
 \rowcolor{goodrow}    4 &   10 \\
\hline
\end{tabular}

\medskip

% LINE THREE TABLE ONE
\begin{tabular}[]{|c|c|c|c|c|c|c|}
\hline
 \rowcolor{goodrow}  11 \\
\hline
 \rowcolor{goodrow}  \cellcolor{yellow}8  \\
\hline
  \rowcolor{goodrow} 8 \\
\hline
\rowcolor{goodrow}  11 \\
\hline
 $\infty$ \\
\hline
 \rowcolor{goodrow}   10 \\
\hline
\end{tabular}

\bigskip

Рассмотрим матрицу смежности графа. \\

Возьмем реберо (1,2)  с наименьшим весом 3 и исключим из таблицы колонки 1, 2. 
Будем искать наименьшее ребро исходящее из вершин 1 или 2.  \\
Это ребро (2,6) с весом 4. Добавим его в дерево и исключим из таблицы колонку 6. Найдем наименьшее ребро исходящее из вершин 1, 2 или 6. \\
Это ребро (6,4) с весом 3. Добавим его в дерево и исключим из таблицы колонку 4. Найдем наименьшее ребро исходящее из вершин 1, 2 ,6 или 4. \\
Это ребро (4,3) с весом 3. Добавим его в дерево и исключим из таблицы колонку 3. Найдем наименьшее ребро исходящее из вершин 1, 2, 6, 4, 3.\\
Это ребро (2,5) с весом 8. Добавим его в дерево и исключим из таблицы колонку 5. \\

Следовательно кратчайшее связывающее дерево
$E(T) = \{ (1,2), (2,6), (6,4), (4,3),(2,5) \} $ \\
Вес дерева 
$f(T) = \sum_{(i,j) \in E(T)} d_{ij}  = \sum_{i=1}^{6} \lambda_i = 17 $

\bigskip
\bigskip

В графе с удвоенным числом ребер дерева \\
$ E(T) || E(T) = \{ 
 (1,2), (2,6), (6,4),(4,3),(2,5)
,(2,1), (6,2), (4,6),(3,4),(5,2)
\}$ \\
Построим Эйлеров цикл $\mu = (1,2,5,2,6,4,3,4,6,2,1)$. \\
Удалим повторения вершин в Эйлеровом цикле для получения приближенного решения $v_1 = (1,2,5,6,4,3,1)$. 

\smallskip 

Вес полученного гамильтонова цикла равен 
$f(v_1) = 3 + 8 + 10 + 3 + 3 + 9 = 36 $

Вычислим относительную точность полученного решения
$ \epsilon = \frac{f(v_1) - f(v_0)}{f(v_0)} = \frac{36 - 30}{30} = 0.2 $

\smallskip

Таким образом найдено приближенное решение задачи коммивояжора.

\bigskip

5. \quad Найдем приближенное решение задачи коммивояжора $v_2$ с помощю второго алгоритма Кристофидеса.
Кратчайшее связывающие дерево имеет ребра
$E(T) = \{ (1,2), (2,6), (6,4), (4,3),(2,5) \}  $

В дереве четыре вершины нечетной степени 1,2,5,3. \\
В оптимальное паросочетание $E(M) = \{(2,1), (5,3)\}$ входит два ребра.
Строим Эйлеров цикл в графе со множеством ребер \\
$E(T) || E(M) = \{ (2,1),(5,3),(1,2),(2,6), (6,4), (4,3),(2,5) \} $

\smallskip 

Полученный эйлеров цикл является одновременно и гамильтоновым
$v_2 = \mu = (1,2,5,3,4,6,1) $ \\
Вес цикла $f(v_2) = 30$. \\
Относительная точность решения $ \epsilon = \frac{f(v_1) - f(v_0)}{f(v_0)} = \frac{30 - 30}{30} = 0$ \\

\smallskip

Таким образом вторым алгоритмом найдено точное решение.



\end{document}
