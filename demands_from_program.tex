

%=========================================
\subsection{Требования к функциональным характеристикам}
\subsubsection{Состав выполняемых функций}
\begin{my_enumerate}
\item Рендеринг данных из несколькох распространенных форматов для хранения 3-х мерных моделей.
\item Возможность изменять положение, ракурс и расширение камеры внутри OpenGL.
\item Проигрывание анимации из файла.
\item Возможность изменять положение и ориентацию модели.
\item Отображение информации о загруженном файле в том числе и колличество треугольников, нормалей, вершин. Также должны быть отображенны объекты типа кости и меш, и реляционные связи между ними. 
\item Отрисовка костей.
\item Изменение параметров проигрывания анимации (скорость, время, повтор фреймов)
\item Отображение информации о текстурах и UV координатах.
\item Отображение информации о материалах загруженной сцены.
\item Отображение интерполированных позиций в между ключевыми точками анимации.
\item Сохранение текущего окна в формате *.png
\end{my_enumerate}

\subsubsection{Организация входных и выходных данных}
\begin{my_enumerate}
\item Входными данными для программы являются файлы созданные  либо вручную (в случае форматов типа NFF), либо экспортированные из пакета 3-х мерного моделлирования (примерами являются Blender,Maya,Cinema4D).
\item Из-за огромного колличества форматов для описания геометрических, объектных и анимационных данных, поддерживать их все не представляется возможным. Поэтому программа должна работать только с огранниченным набором форматов, а именно: Collada (.dae), Neutral File Format (.nff), Stanford  Polygon Library (.ply), Wavefront Object ( *.obj ).
\item Пользователь должен иметь возможность вводить/модифицировать следующие входные данные:
\begin{my_enumerate}
\item Время для которог надо отобразить анимацию.
\item Положение/ориентация модели в OpenGL.
\item Модификаторы для использования при отрисовке модели и костей.
\item Скорость проигрывания анимации.
\item Положение/ракурс камеры в OpenGL.
\end{my_enumerate}
\item Выходные данные для программы это отображение на экране.
\end{my_enumerate}

\subsubsection{Прочие требования}
Других требований к функциональным характеристикам не предъявляется. 

%=========================================
\subsection{Требования к временным характеристикам}
Требования к временным характеристикам программы не предъявляются.


%=========================================
\subsection{Требования к интерфейсу}
Интерфейс должен соответствовать схеме интерфейса, указанной в приложении.


%=========================================
\subsection{Требования к надежности}
\subsubsection{Обеспечение устойчивого функционирования программы}
Программа не должна вне зависимости от входных данных или действий оператора завершатся аварийно. При некорректно введенных параметрах пользователю должно отображаться сообщение об ошибке внутри окна ввода около поля (или группы полей), в которое(-ые) было введено некорректное значение.
\subsubsection{Время восстановления после отказа}
Требования к восстановлению после отказа не предъявляются.
\subsubsection{Отказы из-за некорректных действий оператора}
При попытке запуска алгоритма при не всех введенных данных или данных введенных некорректно, пользователю должно выдаваться сообщение в окне MessageBox.


%=========================================
\subsection{Требования к условиям эксплуатации}
\subsubsection{Вид обслуживания}
Приложение не требует каких-либо видов обслуживания.
\subsubsection{Численность и квалификация персонала}
Минимальное количество персонала, требуемого для работы программы: 1 оператор. Пользователь программы должен знать следующие понятия из линейной алгебры, программирования и 3-х мерного моделирования: вектор, кватернион, матрица направляющих косинусов, меш (англ. mesh), коренная вершина (англ. root node).


%=========================================
\subsection{Требования к составу и параметрам технических средств}
Для оптимальной работы приложения необходимо учесть следующие системные требования:
\begin{my_enumerate}
\item Компьютер, оснащенный:
    \begin{my_enumerate}
    \item 32-разрядный (x86) или 64-разрядный (x64) процессор с тактовой частотой 1 гигагерц (ГГц) или выше;
    \item 512 мегабайт (ГБ) оперативной памяти (ОЗУ);
    \item 1 гигабайт (ГБ) (для 32-разрядной системы) или 2 ГБ (для 64-разрядной системы) пространства на жестком диске;
    \item графическое устройство OpenGL с драйвером версии 3.1 или выше.
    \item видеоадаптер super VGA с расширением 800*600 либо более высоким.
    \end{my_enumerate}
\item Монитор
\item Видеокарта
\item Мышь
\item Клавиатура
\end{my_enumerate}


%=========================================
\subsection{Требования к информационной и програмной совместимости}
Исходный код программы обязательно должен быть написан с использованием языка C\#. Приложению необходим компьютер с поддержкой OpenGL версии не менее 3.1. Операционная система Windows 7 или более поздняя версия Windows. Должен быть установлен .NET Framework версии не ниже 2.0.

