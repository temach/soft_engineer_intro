

%=========================================
\subsection{Требования к функциональным характеристикам}
\subsubsection{Состав выполняемых функций}
\begin{my_enumerate}
\item Чтение данных из распространенного формата коллада для хранения 3-х мерных моделей.
\item Возможность изменять положение и ракурс камеры в OpenGL.
\item Поддержка двух видов камер в OpenGL, первый вид это камера движение которой сковано орбитой вокруг модели и другой тип это камера двигающаяся совершенно свободно.
\item Возможность перейти к любому моменту времени в анимации.
\item Последовательно воспроизведение анимации на экране.
\item Возможность передвигать камеру во время анимации.
\item Вкл./Выкл. отрисовки нормалей данной модели.
\item Вкл./Выкл. отрисовки материала данной модели.
\item Возможность изменять положение модели.
\item Отображение информации об объектах типа кости и меш, и реляционные связи между ними. 
\item Отрисовка всех костей. Отрисовка должна производиться поверх модели.
\item Подсветка отдельных костей выбранных пользователем.
\item Изменение параметров проигрывания анимации, а именно времени, также возможность проиграть в обратную сторону последний интервал между ключевыми фреймами.
\item Анимация должна автоматически запускаться заново после отображения последнего ключевого фрейма.
\end{my_enumerate}

\subsubsection{Организация входных и выходных данных}
\begin{my_enumerate}
\item Входными данными для программы являются файлы  созданные и экспортированные из пакета 3-х мерного моделлирования (примерами таких пакетов являются Blender, Maya, Cinema4D).
\item Из-за огромного колличества форматов для описания геометрических, объектных и анимационных данных, поддерживать их все не представляется возможным. Поэтому программа должна работать только с форматом Collada (.dae).
\item Созданна сцена должна содержать один меш, один трэк анимации и один скелет связанный с мешем.
\item Пользователь должен иметь возможность модифицировать следующие входные данные в ходе работы программы:
\begin{my_enumerate}
\item Время для которог надо отобразить анимацию.
\item Положение/ориентация модели в OpenGL.
\item Модификаторы для использования при отрисовке модели и костей.
\item Положение/ракурс камеры в OpenGL.
\end{my_enumerate}
\item Выходные данные для программы это отображение на экране.
\end{my_enumerate}

\subsubsection{Прочие требования}
\begin{enumerate}
\item Приложение должно вести список недавно открытых файлов.
\item Должно поддерживаться изменение размеров окна приложения, без изменения соотношения проекции OpenGL.
\end{enumerate}

%=========================================
\subsection{Требования к временным характеристикам}
\begin{enumerate}
\item Задержка между кадрами отриванными на экране не должна превышать 0.1 секунд для моделей составленных из менее чем $2,000,000$ треугольников, $15$ костей и  $1,500,000$ вершин.
\end{enumerate}


%=========================================
\subsection{Требования к интерфейсу}
Интерфейс должен обладать шкалой времени для выбора момента времени для анимации.

%=========================================
\subsection{Требования к надежности}
\subsubsection{Обеспечение устойчивого функционирования программы}
Программа не должна вне зависимости от входных данных или действий оператора завершатся аварийно. При некорректно введенных параметрах пользователю должно отображаться сообщение об ошибке.
\subsubsection{Время восстановления после отказа}
Требования к восстановлению после отказа не предъявляются.
\subsubsection{Отказы из-за некорректных действий оператора}
Требования к отказу из-за некорректных действий оператора не предъявляются.

%=========================================
\subsection{Требования к условиям эксплуатации}
\subsubsection{Вид обслуживания}
Приложение не требует каких-либо видов обслуживания.
\subsubsection{Численность и квалификация персонала}
Минимальное количество персонала, требуемого для работы программы: 1 оператор. Пользователь программы должен обладать базовыми знаниями следующих понятий из линейной алгебры, программирования и 3-х мерного моделирования: вектор, кватернион, матрица направляющих косинусов, меш (англ. mesh), кость, корневая вершина (англ. root node).


%=========================================
\subsection{Требования к составу и параметрам технических средств}
Для оптимальной работы приложения необходимо учесть следующие системные требования:
\begin{my_enumerate}
\item Компьютер, оснащенный:
    \begin{my_enumerate}
    \item Обязательно 64-разрядный (x64) процессор с тактовой частотой 1 гигагерц (ГГц) или выше;
    \item 512 мегабайт (ГБ) оперативной памяти (ОЗУ);
    \item 2 ГБ (для 64-разрядной системы) пространства на жестком диске;
    \item графическое устройство OpenGL с драйвером версии 3.1 или выше.
    \end{my_enumerate}
\item Монитор
\item Видеокарта
\item Мышь
\item Клавиатура
\end{my_enumerate}


%=========================================
\subsection{Требования к информационной и програмной совместимости}
Исходный код программы обязательно должен быть написан с использованием языка C\#. Приложению необходим компьютер с поддержкой OpenGL версии не менее 3.1. 64-битная операционная система Windows 7 или более поздняя версия Windows. Должен быть установлен .NET Framework версии не ниже 4.5.1, а также библиотеки Assimp версии не ниже 3.1 и OpenTK версии не ниже 1.1.4

