

%=========================================
\subsection{Требования к функциональным характеристикам}
\subsubsection{Состав выполняемых функций}
\subsubsection{Организация входных и выходных данных}
\subsubsection{Прочие требования}


%=========================================
\subsection{Требования к временным характеристикам}
Требования к временным характеристикам программы не предъявляются.


%=========================================
\subsection{Требования к интерфейсу}
Интерфейс должен соответствовать схеме интерфейса, указанной в приложении.


%=========================================
\subsection{Требования к надежности}
\subsubsection{Обеспечение устойчивого функционирования программы}
Программа не должна вне зависимости от входных данных или действий оператора завершатся аварийно. При некорректно введенных параметрах пользователю должно отображаться сообщение об ошибке внутри окна ввода около поля (или группы полей), в которое(-ые) было введено некорректное значение.
\subsubsection{Время восстановления после отказа}
Требования к восстановлению после отказа не предъявляются.
\subsubsection{Отказы из-за некорректных действий оператора}
При попытке запуска алгоритма при не всех введенных данных или данных введенных некорректно, пользователю должно выдаваться сообщение в окне MessageBox.


%=========================================
\subsection{Требования к условиям эксплуатации}
\subsubsection{Климатические условия}
Условия соответствующие условиям эксплуатации современных компьютеров.
\subsubsection{Вид обслуживания}
Приложение не требует каких-либо видов обслуживания.
\subsubsection{Численность и квалификация персонала}
Минимальное количество персонала, требуемого для работы программы: 1 оператор. Пользователь программы должен знать следующие понятия из линейной алгебры, программирования и 3-х мерного моделирования: вектор, кватернион, матрица направляющих косинусов, меш (англ. mesh), коренная вершина (англ. root node).


%=========================================
\subsection{Требования к составу и параметрам технических средств}
Для оптимальной работы приложения необходимо учесть следующие системные требования:
\begin{my_enumerate}
\item Компьютер, оснащенный:
    \begin{my_enumerate}
    \item 32-разрядный (x86) или 64-разрядный (x64) процессор с тактовой частотой 1 гигагерц (ГГц) или выше;
    \item 512 мегабайт (ГБ) оперативной памяти (ОЗУ);
    \item 1 гигабайт (ГБ) (для 32-разрядной системы) или 2 ГБ (для 64-разрядной системы) пространства на жестком диске;
    \item графическое устройство OpenGL с драйвером версии 3.1 или выше.
    \item видеоадаптер super VGA с расширением 800*600 либо более высоким.
    \end{my_enumerate}
\item Монитор
\item Видеокарта
\item Мышь
\item Клавиатура
\end{my_enumerate}


%=========================================
\subsection{Требования к информационной и програмной совместимости}
Исходный код программы обязательно должен быть написан с использованием языка C\#. Приложению необходим компьютер с поддержкой OpenGL версии не менее 3.1. Операционная система Windows 7 или более поздняя версия Windows. Должен быть установлен .NET Framework версии не ниже 2.0.

